% !TEX root =../main.tex
\subsection{Learning Graphical Models}
\label{sec:learning}
In what follows, we consider the problem of learning the parameters $\thetav$ of an MRF $P(\xx;\thetav)$. Before presenting learning methods, it is instructive to discuss why we want to learn models in the first place (it is generally good practice in life to understand why you are doing something before actually doing it). 
In many domains, the main interest in learning is to {\em understand} the statistical behavior of a set of random variables.
For example, one may wish to find the minimum set of variables that can directly explain a specific variable is (i.e., its neighbors in the MRF graph), or which variable has the greatest effect on another variable, etc. Such questions
usually translate to questions about the structure of the MRF, namely what 
is the underlying graph, or equivalent which $\theta_{ij}$ are close to zero. Methods for learning the graph are known as structure learning, and will be discussed further later. 

%\atodo{Not sure about this argument (commented below). If you just care about Markov blanket, you don't need an accurate description of the distribution. What do you want to say here?}
%For such a purpose, learning an accurate description of $P(\xx)$ is a desirable  goal of learning\footnote{In structure learning this sums down to the difference between the two edges sets $ \min_{E} |\exact{E}\setminus (\exact{E} \cap E)| + |E \setminus (\exact{E} \cap E)|$.

Estimating the actual values of the parmaeters $\theta_{ij}(x_i,x_j)$ is usually not a well defined task, since in the standard MRF parameterization, multiple 
parameter values can correspond to the same distribution.\footnote{Formally, two parameters $\thetav \neq \tilde{\thetav}$, while $P(\xx; \thetav) = P(\xx; \tilde{\thetav}), \forall \xx$} This property is known as re-parameterization and it plays an important role in Belief Propagation (see \secref{sec:belief})

We turn to one of the main goals of learning models, which is answering probabilistic queries. When querying a model, one would like the answer to be as similar as possible to the answer under the true underlying distribution.
For example, if one learns a model of diseases and symptomps, we would like the
probability of a disease given a set of symptoms to be similar in real life and under the model. Note that from this perspective, we don't care whether the graph 
structure or parameters are {\em close} to the real model. 

In a world with no computation limitations the two objectives of learning the exact model and preforming exact inference coincide.\atodo{Explain what you mean in a footnote} Unfortunately, exact inference is NP-hard in the general case, and one is forced to use approximate inference (see \secref{sec:approx}).
When our goal is to correctly answer queries, any learning procedure must take into account the approximate inference to be used on the learned model.

Most of the research on learning graphical models seeks to reconstruct the true underlying model, under some metric. I will next provide a short survey of such approaches, first discussing structure learning and then parameter estimation.
%First is structure leraning, namely learning the edges of the graph $E$,
%then learning the parameters of the model $\thetav$.
\subsection{Structure learning}
\label{sec:structure}
Given a distribution $P(\xx;\exact{\thetav})$ as in \eqref{eq:basic_model},
the structure learning problem is to find the exact edge set $\exact{E}$, given a data set of $\nsamples$ IID samples from $P(\xx;\exact{\thetav})$.

I will start by presenting the sample and time complexity of the problem stated above. 
If the graph is known to be {\em some} tree, the exact structure can be found by maximum spanning tree, where the weights are the empirical mutual information between any two vertices. This is the classic algorithm of Chow and Liu (see \cite{chowLiu}). Later, \cite{tan2011large} proved that sample complexity for this method is $\Omega(\log({\nvars}))$%%U
%\todo{Explain in footnote what sample complexity is in this case}
\footnote{\red{The definition of sample complexity in structure learning is as follow. Let $\nvars$ be the number of variables, $E$ be the result of the algorithm for a data-set of size $\nsamples$, and $\exact{E}$ be the true edge set. The algorithm has sample complexity of $g$, if it guarantees to return $E^*$ with high probability when the number of samples $\nvars$ is bigger that $g$}.}.
%%
In the general case, there seem to be four parameters governing the complexity of structure learning.
The first is the number of vertices $\nvars$.
The second is the maximum degree of the graph, which we denote by $\dmax = \max_{i \in V} d_i$.
The third and the fourth parameters quantify the strength of the interaction between the variables.

In the Ising model\footnote{The Ising model is a pairwise binary model where $\cX = \{-1,1\}$ and $P(\xx;\thetav) = \frac{1}{Z(\thetav)} \exp{\sum_{i \in V}\theta_i + \sum_{ij}\theta_{ij}x_ix_j}$}, these last two parameters are easily expressed; for example, in \cite{santhanam2012information} the authors define $\lambda =
\min_{ij} |\theta_{ij}|$ and $\omega = \max_i \sum_{j \in \nei{i}} |\theta_{ij}|$.
%%U
\red{
However, in other models (where the cardinality of each variable is greater than two, or for non pairwise models), it is unknown how to express these quantities - the strength of the interaction between the variables as a function of the model parameters.
Hence bounds on non-binary models include parameters that restrict the difference between probabilities instead of model parameters.
}%\todo{I don't understand this last sentence. Rephrase.}
%%
Given the above parameters, the sample complexity lower bounds are $\nsamples > \Omega(\dmax \log \nvars)$\cite{bresler2008reconstruction} in the general case, and for the Ising model $\nsamples > \Omega(\max\{\dmax^2, \lambda^{-2}\}\log \nvars)$\cite{santhanam2012information}\footnote{Note that the bound is looser in the general case. This is a result of dropping some model dependent factors - these factors may depend on $d$. See \cite{bresler2008reconstruction} for details}.
%%U
%\todo{Also, this looks like a lower bound, but it’s actually an upper bound so clarify.}
\red{
  Lower bound in this context means, that any algorithm given less samples than the bound cannot insure success with high probability.
  }
%%
Lower bound on time complexity of $\Omega(\nvars^{\frac{\dmax}{2}})$ were given by \cite{bresler2014structure} for statistical algorithms \cite{feldman2013statistical}\footnote{Algorithms using an oracle that returns the expectation of any function with no direct access to the data.}.

There are many variants of structure learning algorithms, and I will give only a partial list. Moreover, the algorithms differ in what is assumed to be given as input, what is the underlying model, and other assumptions, so direct comparison is not always possible. In the following, $a$ will denote parameters (other than $\nvars$, $\dmax$) that depend on the model, usually expressing bounds on interaction strength.

When the MRF involves continuous variables, and corresponds to Gaussian multivariate distributions, the problem becomes easier, since it is equivalent to the estimation of the non-zeros cell in the inverse of the covariance matrix. Examples of algorithms solving this problem are \cite{meinshausen2006high, yuan2007model, friedman2008sparse}. 

The maximum degree $\dmax$ plays an important role in structure learning.
%\todo{Why just lower bounds. Not upper as well?}
Many algorithms \cite{bresler2008reconstruction, abbeel2006learning} assume $\dmax$ is provided as input to the algorithm.
Knowing this quantity, an exhaustive search over all subsets of size $\dmax$ (or a function of it) is preformed, in order to find each vertex's neighbors.
In \cite{bresler2008reconstruction} the time complexity of the algorithm is $O({\nvars}^{2\dmax+1}\log \nvars)$ and sample complexity is $O(\dmax a \log \nvars)$.
In contrast, in \cite{abbeel2006learning} the algorithm does not guarantee reconstructing $E$, but rather similarity in Kullback- Leibler divergence to the true distribution.
The time complexity in this case is $O({\nvars}^{a \dmax})$ and sample complexity $O(\dmax a \log {\nvars})$. Another paper in this class is \cite{wu2013learning} in which, instead of the max degree as an input, they require two quantities that bound the cardinalities of sets which guarantee detection of dependencies. 
%The quantities are the maximum cardinality of two groups. The first $D_1$ insure that for all pairs of vertices that there is no edge between them there exist a set of cardinality less than $D_1$ such given this set the information between the two variables are less than $\epsilon$. The other quantity $D_2$ say exactly the apposite, for all edges there exist a set of cardinality less that $D_2$ such that the information     Conditioning on some sets with size smaller or equals to the given quantities insure dependence . 
\ignore{
\atodo{What is the goal of this paragraph? - explain Wu quantities}
The quantities $D_1$ and $D_2$ are defined as follows: for all $ij \not \in E$ the following take place: $\exists S \subset V, |S|< D_1, \forall T \subset V, |T|<D_2$ $I(X_i;X_j | X_T, X_S)<\epsilon$ and for all $ij \in E$ $\exists T \subset V, |T|< D_2, \forall S \subset V, |S|<D_1$ $I(X_i;X_j | X_T, X_S)>4\epsilon$\footnote{In the Ising model $D_1 = \dmax$ and $D_2 = \dmax -1$. When allowing factors with higher cardinality, two variables may be independent (and any test will indicate they are so), but given another variable they may be dependent - there exists an edge between the two variables. These two quantities bound the size of such sets.}. 
}

Correlation decay is an important concept in structure learning.
There are several definitions of this property, and we focus on the one in \cite{montanari2009graphical}.
Let $B_{\tau}(i)$ denote all the vertices that are within distance (for which a formal definition will be given later) $\tau$ from vertex $i$, and denote by $\partial B_{\tau}(i)$ the boundary of that group.
The assignment of $\partial B_{\tau}(i)$ is required to have negligible effect on the probability of variable $x_i$ for all $i \in V$.
It is now clear, that if the learned model has this characteristic, when deciding what the neighbors of vertex $i$ are, only a subset of vertices need to be considered.
By adding this condition on the true model, \cite{bresler2008reconstruction} improved its time complexity to $O(\dmax^{a\dmax}\nvars+\dmax\nvars^2\log\nvars)$\atodo{what is its here?} .
If it is further assumed that $B_{\tau}(i)$ is a tree (note, the whole graph is not necessarily a tree), then greedy algorithms such as \cite{netrapalli2010greedy, anandkumar2013learning} may be applied.
In \cite{anandkumar2013learning}, the authors preform a greedy algorithm with time complexity of $O(\dmax^{2\dmax}{\nvars})$ and sample complexity of $O(\dmax^2\log \nvars)$, but focus only on the pairwise case.  

In \cite{ravikumar2010high} a different approach is considered. Their algorithm is based on the optimization of the conditional likelihood of a vertex, given all the other vertices with $l_1$ regularization - $\max_{\thetav} \frac{1}{\nsamples}\sum_m \log P(x^{m}_i| \xx^m_{\setminus i};\thetav) - \lambda \sum_{ij} |\theta_{ij}|$.
Given the optimal $\thetav$, the neighborhood set is defined by $\nei{i} = \{\theta_{ij} : |\theta_{ij}| > \alpha\}$, where $\alpha$ is a threshold based on the graph parameters.
Under certain conditions, the algorithm reconstructs $E$ with high probability, in time complexity of $O({\nvars}^4)$ and sample complexity of $O( \dmax^3 \log {\nvars})$.
Note that this algorithm is not very sensitive to the selection of $\alpha$ - and furthermore, the authors suggest how to approximate it.
The algorithm is also very practical.
Its main drawback, is that the conditions (or parameters) that guarantee its success are not directly related to the model parameters, and hence it is very hard to compare it to other such algorithms. Another problem is the assumption of pairwise interaction.     

%%U
\red{
Focusing on Ising models,  \cite{bresler2015efficiently} give a time-efficient algorithm with time complexity of $O(p^2\log p)$ but sample complexity of $O( f(a,\dmax) \log p)$, where $f(a,\dmax)$ is a function which involves the interaction strength and it depends doubly-exponentially in $\dmax$.
%As previous algorithms, it assume a parameter, which lower bound the interaction strength, is given to the algorithm.
Since, in order to ensure correct results, all the samples must be used, the time complexity is the maximum between the time and sample complexity, as in all learning algorithms. 
% of the when a threshold (the minimum difference of the probability when one of its neighbors changes its assignment)
It is interesting that by a greedy algorithm, structure reconstruction could be ensured.
Their algorithm has two stages: First it collects a pseudo-neighborhood for a specific vertex - a set that includes the correct neighborhood.
Then by iterating over all this set, it removes vertices that do not change the conditional probability.
Note that this algorithm cannot be extended to non-pairwise models, which may imply that only in Ising models could greedy algorithm work.
}
%\todo{The definition in brackets is completely unclear, and also unclear how it relates to previous methods you discussed}
%%

To summarize: in most of the above approaches, the maximum degree must be provided as input, and plays a key role in sample and algorithmic complexity.
In most such cases, the sample and time complexity are very close to the known lower bounds.
%%U
\red{
%Hence, the next challenge is to define the necessary assumptions in order to allow efficient structure learning.
%\todo{Not sure what this last sentence means in this context. - i delete it}
The following question is in order: what are the necessary conditions to allow time-efficient (not exponential in $\dmax$) learning?
Correlation decay was considered as an option of such a condition in \cite{montanari2009graphical}, but \cite{bresler2014structure} shows that it, in fact, is not a necessary one.}
%%
Another problem (which is sometimes ignored) is the additional step-up in difficulty when the factor size is not bounded (or is as large as $\dmax$). 
I believe that the conditions under which a model with a bounded factor size and an unbounded degree $\dmax$ can be learned efficiently\footnote{Not exponential in $\dmax$}, may constitute a very interesting research direction.


\subsection{Parameters Learning}
As in the previous section, I will start with the problem definition. 
In parameters learning, it is assumed that the structure is given, although not necessarily the exact one.
The most trivial objective will be to find $\exact{\thetav}$.  
This objective is not well-defined, since two parameters may differ $\hat{\thetav} \neq \tilde{\thetav}$ but still induce the same probability $\forall\ \xx \in \cX^p\ P(\xx;\hat{\thetav}) = P(\xx;\tilde{\thetav})$.
Hence, the objective is finding a parameter that induces the same probability: given a set of i.i.d observations, sampled from $P(\xx;\exact{\thetav})$, find $\thetav$ such that for all input $\forall \xx \in \cX^{p}$ the two return the same (or as close as possible) probability $P(\xx;\thetav) = P(\xx;\exact{\thetav})$.
This, however, is not practical.
First, $P(\xx;\exact{\thetav})$ is not given
Moreover, not all realization of $\xx$ could be expected to be estimated by the sample - and to even check this requirement is exponential to the size of $\xx$.
So another objective, or another way to achieve this goal, is required.
I will start with the  most intuitive approach - maximum likelihood(ML).

\subsubsection{Maximum Likelihood}
\label{sec:max_likelihood}
In ML, one is looking for the parameters that will maximize the probability to sample the observed data-set.
Note, that simply giving probability of $\frac{1}{N}$ to each of the samples is the maximum likelihood solution, but it suffers from extreme overfitting.
Therefore some regularization is needed, and in graphical models this is usually done by forcing a certain structure (thereby forcing independence assumptions) and a bound on the size of the factors.
I will now define the maximum likelihood objective and then discuss its merits and drawbacks.

It is more convenient to maximize the log of the likelihood and to normalize it by the sample size $N$; this does not change the parameters that maximize the function, since $\arg \max_x\frac{1}{N}\log(f(x)) = \arg\max_x f(x)$. 
Denote the data-set by $X = [\xx^1, \ldots, \xx^N]$.
With this, the ML objective can be formally defined by using the model in \eqref{eq:basic_model} as, 
\bean
\likelihood{X;\thetav} &=& \frac{1}{N}\log\left(\prod_{n=1}^N p(\xx^n;\thetav)\right) \nonumber \\
&=& \frac{1}{N}\sum_{n=1}^N\log\left(p(\xx^n;\thetav)\right)\nonumber\\
&=& \frac{1}{N}\sum_{n=1}^N\log\left(\frac{1}{Z(\thetav)} \exp{\sum_{ij \in E} \theta_{ij}(x^n_i,x^n_j) + \sum_{n \in V} \theta_i(x^n_i)}\right)\nonumber\\
&=& \frac{1}{N}\sum_{n=1}^N \sum_{ij \in E} \theta_{ij}(x^n_i,x^n_j) + \sum_{i \in V} \theta_i(x^n_i) - A(\thetav)\nonumber\\
&=& \sum_{ij \in E} \sum_{x_i,x_j \in \cX}\mub_{ij}(x_i,x_j)\theta_{ij}(x_i,x_j) + \sum_{i \in V} \sum_{x_i \in \cX}\mub_i(x_i)\theta_i(x_i) -A(\thetav)\nonumber\\
&=&  \mubv \cdot \thetav - A(\thetav) \label{eq:ML}
\eean
where $\mubv$ is the empirical marginals
\be
\label{eq:empirical_mar}
\mub_i(x_i) = \frac{1}{N}\sum_{n=1}^{N} \deltaF{x_i=x_i^n },\ \
\mub_{ij}(x_i,x_j)  = \frac{1}{N}\sum_{n=1}^{N} \deltaF{x_i=x_i^n \land  x_j=x_j^n}
\ee
ordered in a vector in the same way as $\thetav$.

Since the $\thetav$ that maximizes ML is sought,
taking the derivative and comparing to zero is the called-for step:
\bean
\label{eq:ml_derv_single}
\frac{ \partial \likelihood{X;\thetav}}{\partial \theta_{i}(x_i)} &= \mub_{i}(x_i) - \frac{ \partial \log(Z(\thetav))}{\partial \theta_{i}(x_i)} =& \mub_{i}(x_i) - p(x_i;\thetav) \\
\label{eq:ml_derv_pairs}
\frac{ \partial \likelihood{X;\thetav}}{\partial \theta_{ij}(x_i,x_j)} &= \mub_{ij}(x_i,x_j) - \frac{ \partial \log(Z(\thetav))}{\partial \theta_{ij}(x_i,x_j)} =& \mub_{ij}(x_i,x_j) - p(x_i,x_j;\thetav)
\eean
 where the derivative of the partition function is given by \eqref{eq:pratition_derivative}.
It may be concluded, that the ML parameter $\ml{\thetav}$ must satisfy the moment matching characters - the marginals of the model must equal to the empirical marginals.
\be
\label{eq:moment_matching}
\mub_i(x_i) = p(x_i;\thetav), \ \ \ \mub_{ij}(x_i,x_j) = p(x_i,x_j;\thetav)
\ee
This feature is very intuitive, as the resulted probability is required to be able to re-generate the sample's marginals.
ML has other desirable qualities (under some conditions): 
It is consistent - when $N \to \infty$, the exact parameters $\lim_{N \to \infty} \arg \max_{\thetav} p(X^N;\theta) =  \exact{\thetav}$ will be recovered\footnote{This does not contradict the above discussion - part of the conditions is uniqueness $\exact{\thetav}$.}. 
It is efficient - at the limit, no other consistent estimator has a lower mean square error than ML.   
Asymptotic Normality - with sufficient data, the mean of the ML estimator will be the true parameter with added Gaussian noise, and the probability to get an erroneous parameter decreases exponentially fast.

It would appear that ML is indeed all we need, however this is not the whole picture,
and ML suffers from several drawbacks.
The first and most important is, that it is not time- efficient.
Calculating the ML is $\#P$ since it involves calculating the partition function.
Moreover, even calculating its derivative is $\#P$ since it involves inferring the marginals of the model.
%So direct gradient descent is not possible since it involve calculating the current model marginals.
Indeed \ignore{Even if we tackle the time complexity of finding the marginals}, results by \cite{bresler2014hardness} and \cite{montanari2015computational} show that if $NP \neq RP$ then no algorithm using only the model marginals (marginals corresponding to the model parameters) can learn the model parameters. 
Hence, if a time- efficient algorithm that finds the ML parameter is to be found, it must use a higher order of marginals.

In practice, since exact inference is hard, approximate inference is used.
Due to BP success, it is used to approximate the marginal in gradient descent.
The next section discusses the meaning of doing so.
\subsubsection{Maximum Likelihood with BP as the approximate inference}
\label{sec:Bethe_ML}
Optimizing ML is usually achieved by gradient descent.
In order to calculate the derivative, the marginals of the current model are needed.
BP is a successful algorithm for finding the marginals,
so using it to find the marginals seems like a good choice.
But what does it optimize exactly?

Using the variation stand point, \eqref{eq:ML} can be written as
\be
\sup_{\thetav \in \Omega^G} \likelihood{X;\thetav} = \sup_{\thetav \in \Omega^G}\left\{ \mubv \cdot \thetav - \sup_{\muv \in \margpoly} \left\{\muv \cdot \thetav - A^*(\muv)\right\}\right\}
\ee
Since BP is used, and remembering the result of \cite{yedidia2000generalized}(\claimref{thm:bp_bethe}), the optimizing ML with BP can be written as ,
\be
\label{eq:bethe_like}
\bethelikelihood{X;\thetav}  = \sup_{\thetav \in \Omega^G} \left\{\mubv \cdot \thetav - \sup_{\tauv \in \lclmargpoly} \left\{\tauv \cdot \thetav + H_B(\tauv)\right\}\right\}
\ee
This will be called the Bethe maximum likelihood.
It is a convex function of $\thetav$, since $A_B(\thetav)$ is a convex function of $\thetav$. (Note that it is indeed convex, but not smooth.)
Moreover, it can be proven that $\bethelikelihood{X; \thetav}$ is \textbf{strictly} convex. % (see \appref{})
To summarize, optimizing ML by using BP as the inference algorithm in gradient descent, will in fact optimize the Bethe likelihood and not the exact one.

Bethe and BP are exact for models with a tree dependency graph, and structure learning is easy for such models.
It is clear that optimizing ML is feasible, but cannot better be done?  
The canonical parameters are a closed form set of equations, that learn the exact parameters for tree models.
They are defined as,
\be
\label{eq:canonical}
\theta_{ij}(x_i,x_j) = \log{\frac{\mub_{ij}(x_i,x_j)}{\mub_i(x_i)\mub_j(x_j)}}, \ \ \ \theta_i(x_i) = \log{\mub_i(x_i)}
\ee
So if the model graph is a tree, then the canonical parameters could be used, and no gradient descent is necessary.

Apparently, the Bethe ML (or a close formulation of it) is hiding in other algorithms, for example \cite{ganapathi2012constrained}.
In this paper, the ML is optimized via its dual - maximizing the entropy with moment matching constraints.
Since this problem is hard, the problem is relaxed in the same way as the Bethe - using the local marginal polytope instead of the marginal polytope, and using the Bethe entropy instead of the exact one; 
so, in fact, trying to optimize an objective that is similar to the Bethe likelihood \footnote{Note that if no other constraints are found, the moment matching condition in the Bethe entropy maximization fully specifies the solution}.
To optimize this relaxed objective, the Bethe free energy must be found - as achieved by \cite{yuille2002cccp}.

But let us return to the optimization of the Bethe ML.
The Bethe likelihood $\bethelikelihood{X;\thetav}$ is a non-smooth function, hence the optimal conditions sum down to,
\be
\label{eq:bethe_opt}
\mu_{ij}(x_i,x_j) = \convexhull{ \arg \sup_{\tauv \in \lclmargpoly} \left\{\tauv \cdot \thetav + H_B(\tauv)\right\}}
\ee
where $\convexhull{[\tauv_1,\ldots,\tauv_k]}$ is the convex hull of $[\tauv_1,\ldots,\tauv_k]$.
It is easy to see that if the Bethe energy has only one maximum, the moment matching condition \eqref{eq:moment_matching} is satisfied.
Moreover, running ML with BP will converge to this result.
In contrast, in the case of multiple maximum, no moment matching is guaranteed - the learned parameters will not reconstruct the empirical marginals.
As a result, running ML with BP will converge slowly but will keep ``jumping'' between the different maximums. 
%As a result, running ML with BP will keep ``jumping'' between the different maximums, and converge only extremely slowly. 
A question remains, do such empirical marginals exist?

In the paper \cite{heinemann2012cannot} we prove that such empirical marginals do exist, and show that there are indeed marginals in the marginal polytope that will \textbf{never} (for any parameter) be the result of BP.
Note that this result also has relevance to the paper of \cite{ganapathi2012constrained} - there are marginals for which the results of the algorithms depend on the initialization of the gradient descent.   

We will now consider an approximation to the ML Pseudo-Likelihood.
\subsubsection{Pseudo-Likelihood}
Pseudo-Likelihood (PL)\cite{besag1975statistical} is a way to approximate the likelihood.
It is the first of a larger family of likelihood approximations called composite likelihood \cite{lindsay1988composite}.
Specifically, noting that the likelihood may be written as $ p(\xx;\thetav) = \prod_{i=1}^p p(x_i |x_1,\ldots x_{i-1})$,
the PL approximation is
\be
pl(\xx;\thetav) = \prod_{i=1}^p p(x_i |\xx_{\setminus i};\thetav) = \prod_{i=1}^p p(x_i |\xx_{\nei{i}};\thetav)
\ee
\ignore{
Using model \eqref{eq:basic_model}, the probability of a vertex given its neighbors can be written as,
\be
\label{eq:pl_condition}
p(x_k |\xx_{\setminus k};\thetav) 
%&=& \frac{p(\xx;\thetav)}{p(\xx_{\setminus k};\thetav)}\\
%&=& \frac{p(\xx;\thetav)}{\sum_{x_k \in \cX }p(\xx;\thetav)}\\
%&=&\frac{\exp{\sum_{ij \in E} \theta_{ij}(x_i,x_j) + \sum_{ i \in V} \theta_i(x_i)}}{\sum_{ \tilde{x}_{k} \in \cX }\exp{\sum_{ij \in E} \theta_{ij}(x_i,x_j) + \sum_{i \in V} \theta_i(x_i)}}\\
%&=&\frac{\exp{\sum_{j \in \nei{k}} \theta_{kj}(x_k,x_j) + \theta_k(x_k)}}{\sum_{\tilde{x}_k \in \cX}\exp{\sum_{j \in \nei{k}} \theta_{kj}(\tilde{x}_k,x_j) + \theta_k(\tilde{x}_k)}}\\
=\frac{\exp{\sum_{j \in \nei{k}} \theta_{kj}(x_k,x_j) + \theta_k(x_k)}}{\sum_{\tilde{x}_k \in \cX}\exp{\sum_{j \in \nei{k}} \theta_{kj}(\tilde{x}_k,x_j) + \theta_k(\tilde{x}_k)}}\\
\ee
}
Normalizing and taking the log, the optimization of PL can be written as,
\bean
\label{eq:pl_maximization}
\max_{\thetav} \frac{1}{N} \log{\left(\prod_{i=1}^Npl(\xx^i;\thetav)\right)}
%&&\frac{1}{N} \log \left(\prod_{n=1}^N \prod_{i \in V} p(x^n_i | \xx^n_{\nei{i};\thetav})\right) \\ 
%&=&\sum_{i \in V}\frac{1}{N}\sum_{n=1}^N \sum_{j \in \nei{i}} \theta_{ij}(x^n_i,x^n_j) + \theta_i(x^n_i)\\
%&&-\log\left(\sum_{\tilde{x}_i \in \cX}\exp{\sum_{j \in \nei{i}} \theta_{ij}(\tilde{x}_i,x^n_j) + \theta_i(\tilde{x}_i)}\right) \\
&=& \sum_{i \in V} \sum_{j \in \nei{i}}\sum_{x_i,x_j \in \cX} \mub_{ij}(x_i,x_j)\theta_{ij}(x_i,x_j) + \sum_{x_i \in \cX}\mub_i(x_i)\theta_i(x_i)\\ 
&&-\sum_{\xx_{\nei{i}} \in \cX^{|\nei{i}|}} \mub_{\nei{i}}(\xx_{\nei{i}})\log\left(\sum_{\tilde{x}_i \in \cX}\exp{\sum_{j \in \nei{i}} \theta_{ij}(\tilde{x}_i,x_j) + \theta_i(\tilde{x}_i)}\right)\nonumber
\eean
And the derivatives \footnote{The pairwise derivatives are similar.} are,
\be
\label{eq:pl_derivative}
%\frac{\partial pl(X^n;\thetav)}{\partial \theta_{ij}(x_i,x_j)} &=& 2 \mub_{ij}(x_i,x_j) \\
%&&- \sum_{\xx_{\nei{i}\setminus j} \in \cX^{|\nei{i}|-1}} \mub_{\nei{i}}(x_j,  \xx_{\nei{i}\setminus j}) p(x_i|x_j, \xx_{\nei{i}\setminus j};\thetav)\\
%&& - \sum_{\xx_{\nei{j}\setminus i} \in \cX^{|\nei{j}|-1}} \mub_{\nei{j}}(x_i,  \xx_{\nei{j}\setminus i}) p(x_j|x_i, \xx_{\nei{i}\setminus j};\thetav)\\
%\frac{\partial pl(X^n;\thetav)}{\partial \theta_{i}(x_i)} &=& 
%\mub_i(x_i) - \sum_{\xx_{\nei{i}} \in \cX^{|\nei{i}|}} \mub_{\nei{i}}(\xx_{\nei{i}}) p(x_i|\xx_{\nei{i}};\thetav)
\frac{\partial pl(X^n;\thetav)}{\partial \theta_{i}(x_i)} =  \mub_i(x_i) - \sum_{\xx_{\nei{i}} \in \cX^{|\nei{i}|}} \mub_{\nei{i}}(\xx_{\nei{i}}) p(x_i|\xx_{\nei{i}};\thetav)
\ee

The first thing to note is that \eqref{eq:pl_maximization} is easy to optimize, since both it and its derivative (\eqref{eq:pl_derivative}) sum over the neighbor variables, not the whole models.
As a consequence, optimizing by gradient descent is plausible.
Second, note that \eqref{eq:pl_derivative} gives us a similar condition to the moment matching condition \eqref{eq:moment_matching} ML.
The difference arises from the fact that only the conditional probability is estimated, hence the need to sum over all neighbor values.
This gives intuition to why PL is less sample- efficient - it needs to approximate not only the factor's marginals but marginals of the factor and its neighbors.
%Unfortunately, for each derivative we not only need to a approximate the pairwise marginals but the marginals of the whole neighborhood.
If one considers the result by \cite{bresler2014hardness} and \cite{montanari2015computational}, this is not surprising - to have any time- efficient algorithm, it is necessary to use higher moments.
Other characteristics of PL, such as $\thetav$ being asymptotically normally distributed, can be derived from the more general family of composite likelihood, see \cite{varin2011overview}.
\ignore{
Another view point of PL is, that instead of approximating the likelihood, it just preforming ML on the conditional marginals. 
We are looking for the parameters that maximize the conditional probability for each vertex separately.
The derivative of the singleton parameter is the same.
While the pairwise change, so it include only one of the summation - hence a more moment matching like equation.
}

The first reason to use PL and not ML is time complexity.
While PL parameters can easily be computed via gradient descent,
no efficient way is known to recover ML parameters.
The question whether there are cases when PL gives better results than ML, remains open.
In the well- specified case, the model structure and factors being exact, it was shown that ML is more efficient (sample wise) than PL, \cite{liang2008asymptotic};
while in the miss- specified case, the more realistic one, it is not clear which method is better (see \cite{varin2011overview} for discussion).
An interesting direction for research is to design an inference algorithm for PL - 
meaning, design an efficient inference algorithm such that it is promised to return the empirical marginals when the model parameter were learned using PL.
This may have great importance, as I will show in \secref{sec:inferning}.

\subsubsection{Other Methods}
Contrastive Divergance (CD) \cite{hinton2002training} was designed for models where part of the variables are hidden (variables not included in any sample), but it could be used on general models.
Recall from \eqref{eq:ml_derv_single} and \eqref{eq:ml_derv_pairs}, that the reason gradient-descent is impractical,  is the hardness of evaluating the marginals according to the parameters $p(x_i,x_j;\thetav)$. 
One way to approximate these marginals is by MCMC (see \secref{sec:sampling}).
Preforming only a small number of steps - usually one - to evaluate $p(x_i,x_j;\thetav)$, results in CD.
Justification to this method was given by \cite{bengio2009justifying} and \cite{carreira2005contrastive}, but its success in learning Restricted Boltzmann Machine (RBM) in fact gains it its importance.

In contrast to the previous methods,  \cite{abbeel2006learning} give a closed form equation for parameter estimation.
Using the ideas from the proof of the Hammersley -Clifford theorem,  it estimates the canonical factors  - hence learning the graph parameters.
More specifically, any distribution can be written (using the model in \eqref{eq:ciluqe_prob}) by 
\be
p(\xx;\exact{\thetav}) = \frac{1}{Z(\thetav)} \exp{\sum_{c \in \cC}\thetav_c(\xx_c)} = p(\tilde{\xx})\exp{\sum_{c\in\cC} \thetav^{\dagger}_c(\xx_c)}
\ee
Now the canonical factors $\thetav^{\dagger}$ are defined by
\be
\thetav^{\dagger}_{c}(\xx_c) = \sum_{b \subseteq c} -1^{|c \setminus b|} \log{p(f(b,\xx_c,\tilde{\xx}))}
\ee
where  $f(b,\xx_c,\tilde{\xx}) : \cI \times \cX^{|c|} \times \cX^{|d|}\to \cX^{|d|}$ such that $b \subseteq c\subseteq d$, and for all $i \in d$ - if $i \in b$ then $x_i = [\xx_c]_i$, otherwise $x_i = \tilde{\xx_i}$ for the fixed assignment $\tilde{\xx}$.
This can be further simplified by using independence 
\be
\label{eq:abbeel}
\thetav^{\dagger}_{c}(\xx_c) = \sum_{b \subseteq c} -1^{|c \setminus b|} \log{p(f(b,\xx_c,\tilde{\xx}_c) |\tilde{\xx}_{\nei{c}})}
\ee
So a factor involves only the clique and the clique's neighbors.
Therefore, to estimate the parameters, one needs to simply change $p(f(b,\xx_c,\tilde{\xx}_c) |\tilde{\xx}_{\nei{c}})$ to the corresponding empirical marginals.
In a later paper, \cite{roy2009learning} further simplifies the above equation, but with no implication to parameter learning.
Note the resemblance of  \eqref{eq:abbeel} to the PL equation.
Indeed, in \cite{bradley2012sample} it is proven that the two methods give the same results;
they include a compression of the sample complexity of this method, PL (more accurately, composite likelihoods) and ML.

Other methods follow along the same lines - avoiding the partition function. 
For example, \cite{hyvarinen2007some} use the ratio between the probability of two assignments, hence canceling the partition function.
Note that using the ratio is basically similar to using the conditional probability, and indeed \cite{marlin2010inductive} and \cite{marlin2012asymptotic} compare the methods to pseudo-likelihood and explain the differences.
Other methods like \cite{mizrahi2014icml} simply cut off each clique with its neighbors, and learn each factor independently\footnote{It assumes independence of the clique neighbors from their other neighbors. However it is not clear how it effects the sample complexity, and non is given in the paper.}.

To conclude, ML is provably the most sample complexity efficient, but no time efficient algorithm is known.
Moreover, it was proven by \cite{bresler2014hardness} and \cite{montanari2015computational} that no algorithm using only the sufficient statistics in the clique order, can optimize ML.
On the other hand, PL is time efficient but its sample complexity is worse than that of ML.
Methods like composite likelihood can give better sample complexity results, while still being time efficient.

\subsection{Inferning}
\label{sec:inferning}
In a world without computation limitation, learning the exact model is always optimal.
In our world, however, it is not always the case.
If the quality of the learning process is with respect to the outcome of the approximate inference algorithm,
then the learned model should be selected such that the inference algorithm will return good results, not such that it be close to the real model.
In \cite{wainwright2006estimating} the authors bring up this point, and empirically demonstrate that when TRW is the inference algorithm, the parameters should indeed be the TRW parameters. 
Note that in this paper, the structure of the model was given.
An interesting question remains, how learning should be preformed when the structure is not given, and the goal is preforming inference on the learned model.

The complexity of inference for a given model is governed by its tree-width, \cite{robertson1983graph}, \cite{robertson1994quickly}.
Learning models with bound tree-width will guarantee easy inference.
This, however, turns out to be a hard problem in itself \cite{srebro2001maximum}.
Many algorithms tried to approximate this problem, for example \cite{karger2001learning}, \cite{bach2001thin}, and \cite{elidan2009learning}.
%
Another class of interesting models is where the inference algorithm is not exact, but still guaranteed to return good approximation. 
As mentioned earlier (see here \ref{sec:belief} ), running BP on models that are tree-like can be proven to return good approximation \cite{dembo2010ising}.
Tree-like models are models that are locally trees, but allow far- away cycles.
The meaning of far- away in models, is translated to negligible correlation between vertices or a vertex and a group, this attribute is called correlation decay (see \secref{sec:structure} for more detailed explanation of this attribute).
\ignore{ $\epsilon$ influence of the assignment of two vertices on each other (or on a group of vertices to a vertex).}
In other words, in models where in the influential surrounding are trees, inference is plausible.
In our paper \cite{heinemann2014inferning} the structure and parameters are learned such the resulted model belong to such model family.
%
Fast mixing models are another model family,  known to have good approximation inference algorithms.
As explained above  \secref{sec:sampling}, if the mixing time is small, sampling from the distribution is feasible -resulting in good approximation inference. 
In the paper \cite{domke2015maximum}, a learning algorithm is presented, that guarantees to return a model which belongs to the fast mixing model family,
thereby guaranteeing the quality of the inferring result.
