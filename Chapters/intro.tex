The world around is digitized.
What we read(news, letters, chats, articles etc), what we see (pictures, movies), what we hear(music, radio, Skype,etc) are already (for several years) saved in the format of bytes.
Factories, medical records, financial action are all also been digitized
But this is only the tip of the iceberg, gradually most of are surrounding will be monitored and be transformed into  bytes.
This amount of data is almost useless to humans - no human can even observe this amount of data not to mention say anything intelligible on it.
But even manually design tools to use this amount of data is not possible.
The need of automated ways to create tools that will handle the growing amount of data is immanent.   

Machine learning, is the discipline that try to close this gap.
It objective is to find algorithms that takes as an input past data and return tools that can monitored, classify, recommend etc based on the current data.
%The challenge is huge but results of this field merit are already exist.
Graphical Models (GM), Support Vector Machine (SVM) and deep learning are the leading methods in machine learning.
In this work i will touch each of these methods and try to contribute to their understanding and success.

Graphical models describe the world as stochastic process.
%Each feature is describe by a vertex which is connect to all features that directly effects it.
Given an event it return the probability to see this event.
This give us a huge power, the most probable event can be found or the probability of each feature assignment cab be calculated.
Moreover, this inference queries could be asked when part of the assignment is seen while querying on the unseen part.
The problem is that inference in GM is usually hard - all known algorithm are impractical\footnote{More formally MAP is NP-complete and MAR is \#P is the general case see \secref{sec:def}}.
Hence approximate inference play important rule in GM research.

As the above imply, the required ability is automatically create  the tools which translate to learning graphical models.


 
