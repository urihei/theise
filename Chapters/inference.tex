% !TEX root =../main.tex
\subsection{Approximate inference}
\label{sec:approx}
In order to use graphical models in practice, one must solve the inference tasks
described in Section \ref{sec:inference}. As mentioned earlier, in the general case
this is an ``intractable'' task (by which we mean it has no known polynomial time algorithm). This has prompted much exciting research on approximate inference algorithm, combining
tools from combinatorial optimization, convex optimization and graph theory. 

Below, we will provide a brief overview of approximate inference approaches, and describe two of the most widespread paradigms: variational methods and sampling based methods.  
We will focus only on the problem of finding the marginals, since it is more relevant for 
the learning problem, which is the focus of this thesis.

\subsection{Variational Methods}
\label{sec:variational_methods}
Variational methods are a large class of approximate inference algorithms. The key
underlying approach in all of these is casting inference as a continuous (as opposed to discrete) optimization problem. Our review below largely follows the excellent and rigorous introduction to the topic in \cite{wainwright2008graphical}. 

The goal of marginal inference is to calculate the marginals of the model $P(\xx;\thetav)$. Here we focus on calculating singleton marginals $\mu_i^{\thetav}(x_i)$ and pairwise marginals  $\mu^{\thetav}_{ij}(x_i, x_j)$ (see \eqref{eq:pratition_derivative}).
In what follows we will denote by $\muv$ the vector constructed from concatenating all these marginals, and denote its dimension by $d$.\footnote{For example, in the case of binary variables we will have $d = 2|V| + 4|E|$.}

The variational approach proceeds by defining some function $g(\muv)$ of the marginals, and maximizing this function, subject to some constraints, to obtain the approximate marginals. A first step to realizing such an approach is to understand which values of $\muv$ are possible. In other words, which vectors $\muv$ are marginals of some model $P(\xx;\thetav)$. Clearly we need not consider vectors $\muv$ that cannot be marginals of any model.

The above point leads to the definition of the ``Marginal Polytope''  \cite{wainwright2008graphical}. The Marginal Polytope, which we denote by $\margpoly^G$,  is a set of $\muv$ vector, which are the marginals of
``some'' distribution. The definition does not require this distribution to be an MRF. However, as shown in  \cite{wainwright2008graphical}, any $\muv$ in the interior of $\margpoly^G$ is in fact also the marginals vector of an MRF $P(\xx;\thetav)$, for some $\thetav$. The formal definition of $\margpoly^G$ is:
\be
\label{eq:margpoly}
\margpoly^G = \left\{ \muv \in [0,1]^d\ \left| 
\begin{array}{lr}
  \exists P(\xx) \ s.t. \\
  \forall i \in V \land \forall x_i \in \cX &   P(x_i) = \mu_i(x_i)\\
  \forall ij \in E \land \forall x_i, x_j \in \cX &P(x_i,x_j) = \mu_{ij}(x_i,x_j)
\end{array} \right. \right\}
\ee

Next, we define a continuous optimization problem whose maximizer will be $\muv^{\thetav}$, and whose maximum value will be the log partition function $A(\thetav)$ (see \eqref{eq:partition_function}). 

Since $A(\thetav)$ is a convex function, it has a conjugate convex function $A^*(\muv)$, defined as follows \cite{boyd2004convex}: 
\be
\label{eq:conjugate_partition}
A^*(\muv) = \sup_{\thetav} \left\{\muv \cdot \thetav - A(\thetav)\right\}
\ee
Since the conjugate of $A^*(\muv)$ is $A(\thetav)$, we can cast $A(\thetav)$ as the following optimization problem:
\be
A(\thetav) = A^{**}(\thetav) = \sup_{\muv} \left\{\muv \cdot \thetav - A^*(\muv)\right\}
\label{eq:conjconj}
\ee
To further simplify this problem, we note several properties of $A^*(\muv)$.
First if $\muv\notin\margpoly^G$ then $A^*(\muv) = \infty$.
Thus, we can limit optimization to $\muv\in\margpoly^G$.
Next, for any $\muv$ there exists a set of MRF parameters $\thetav(\muv)$ such that the marginals of $P(\xx;\thetav(\muv))$ are $\muv$.\footnote{The function $\thetav(\muv)$ is sometimes referred to as the mapping from mean parameters to canonical parameters. See \cite{wainwright2008graphical} for a thorough discussion} Denote the entropy of a distribution $P(\xx)$ by $H(P(\xx))$. Then for $\muv\in\margpoly^G$ the function $A^*(\muv)$ turns out to be $-H(P(\xx;\thetav(\muv)))$.  

The above observations imply that \eqref{eq:conjconj} can be written as:
%Two points should be noted. First, if $\muv$ is in the interior of $\margpoly^G$ \footnote{This applies only for marginals in the inner part of the marginal polytope. For marginals on the boundaries, more delicate treatment should be given} then $A^*(\muv) = -H(P(\xx;\thetav(\muv)))$ the conjugate of the log partition function equals to the entropy of the probability the marginals of are $\muv$.
%Second, \eqref{eq:conjugate_partition} is similar to the maximum likelihood objective (see \secref{sec:max_likelihood}).
%It can be shown that $A(\thetav)$ is a convex function of $\thetav$, which leads to ${A^{*}}^* = A$. Having that, the variational expression of $A(\thetav)$ may be written as, 
\be
A(\thetav) = \sup_{\muv \in \margpoly^G}\left \{ \muv \cdot \thetav + H(P(\xx;\thetav(\muv))) \right\} \label{eq:variation_A} 
\ee
Also, since the gradient of $A(\thetav)$ is $\muv^{\thetav}$, it follows that:
\be
\muv^{\thetav}= \arg \sup_{\muv \in \margpoly^G}\left \{ \muv \cdot \thetav + H(P(\xx;\thetav(\muv))) \right\} \label{eq:arg_variation_A}
\ee
\eqref{eq:arg_variation_A} and \eqref{eq:variation_A} offer an optimization based view of marginals and partition function calculation. They form the starting point for most
variational approximations.


Not surprisingly, solving the above optimization problem is hard, for two reasons. First,  optimizing over the marginal-polytope $\margpoly^G$ is hard in general (even if the objective is linear). Describing $\margpoly^G$ requires a number of inequalities that is exponential in $d$.  Second, both calculating $\muv^{\thetav}$ is hard, as is calculating the entropy of the resulting MRF.

The advantage of the above representation lies in the simplicity of approximating its constraints and objectives, to obtain a more ``managable'' optimization problem.\footnote{We use the loose term managable, since not all approximations correspond to polynomial time solvable optimization problem. For example mean field attempts to solve a non-convex problem and may obtain sub-optimal solutions}

{We next present two well-known variational approximations: mean-field and belief propagation}.
%% Two methods of approximation will next be presented. The first is the mean-field method \cite{peterson1987mean} where both the marginal-polytope and the entropy are restricted to a sub-graph of $G$ such that both are easy to calculate.
%% The second method gives different approximations to the entropy and to the marginal polytope.
%% For example, the local polytope (defined in \eqref{eq:local_polytope}) will approximate the marginal-polytope, while the entropy is approximated by the Bethe-entropy. 
%% The resulting approximation is the Bethe approximation, which is related to the known algorithm Belief-Propagation\cite{pearl1986fusion, yedidia2000generalized}.
\subsubsection{Mean Field}
Mean field methods have a long history, starting from statistical physics, and later popularized in the machine learning community \cite{weiss1907hypothese,peterson1987mean}. There are various ways of deriving them, and here we will provide one based on the variational view of \eqref{eq:variation_A}.\footnote{A popular and earlier derivation, casts mean field as finding a simple distribution that approximates $P(\xx;\thetav)$ in a KL divergence sense.} 
%The term mean field methods (e.g., see \cite{peterson1987mean}) corresponds to approximate inference approaches that that approximate a complex model by a simpler one.
%In inference, it approximates the real model marginals by marginals of a simpler model.

%\atodo{It's incorrect ot say mean field uses $\margpoly^{G_I}$ since that contains only singleton marginals}
We begin by considering all distributions $P(\xx)$ corresponding to a set of $\nvars$ independent variables, as in \eqref{eq:independent}. The pairwise marginals of such distributions will also be independent, namely given by $\mu_{ij}(x_i,x_j) = \mu_i(x_i)\mu_j(x_j)$.
Such distributions are therefore always in $\margpoly^G$.
This implies that the following set $\margpoly^I$ is a subset of $\margpoly^G$ (namely $\margpoly^{I} \subseteq \margpoly^{G}$):
%\footnote{ \red{If $G$ include edges,  $\margpoly^{I} \subset \margpoly^{G}$. Since both sets share the same extreme point $\margpoly^{I}$ is a non-convex set.}}:
%\atodo{Strictly speaking this is not a marginal polytope...}
%More specifically, it replaces the marginal polytope (of the true dependencies graph) and the true entropy, with the marginal polytope and entropy of a simpler dependencies graph.
%I will illustrate this technique by choosing the simplest graph - the independent graph denoted by $G_I$.
%In this case, the marginal polytope sums down to,
\be
\margpoly^{I} = \left\{\muv \in \Re^{d}\left|
\begin{array}{lr}
\forall i \in V,\ \forall x_i \in \cX & \mu_i(x_i) \geq 0 \\
\forall i \in V & \sum_{x \in \cX} \mu_i(x_i) = 1\\
\forall ij \in E,\ x_i,x_j \in \cX & \mu_{ij}(x_i,x_j) = \mu_i(x_i)\mu_j(x_j)
\end{array}
\right.\right\}
\ee
%Note that the dimension of the marginals is with respect to the original graph,
%moreover, $\margpoly^{I} \subseteq \margpoly^{G}$\footnote{Note that marginals corresponding to the pairwise interaction are constrained to reflect independence, a constraint that does not exist in the original marginal polytope.}.\\

We next turn to calculating $A^*(\muv)$ for $\muv\in\margpoly^{I}$. It is easy to see that for any $\muv \in  \margpoly^{I}$  the distribution $P(\xx) = \prod_{i} \mu_i(x_i)$ has pairwise marginals $\mu_{ij}(x_i,x_j) =  \mu_i(x_i)\mu_j(x_j)$. Since $P$ is an MRF, it follows that
$\thetav(\muv)$ are the parameters specifying $P$. Also, the entropy of $P$ is just the sum of individual entropies:
\be
H_{I}(\muv) = -\sum_{i}\sum_{x_i} \mu_i(x_i)\log\mu_i(x_i)
\ee 

Putting all of the above together we have the mean field approximation of the partition function:  
\be
A_{I}(\thetav) = \sup_{\muv \in \margpoly^{I}}\left \{ \muv \cdot \thetav + H_{I}(\muv))) \right\} \label{eq:naive_mean_field}~,
\ee
Note that $A(\thetav) \geq A_{I}(\thetav)$ since by restricting the marginal polytope the result can only decrease, and $A^*(\muv)$ is exact. However, the  function maximized in \eqref{eq:naive_mean_field} is non-convex, so we cannot generally find the above maximum efficiently.\footnote{Note that for any feasible $\muv$ the right hand side is still a lower
bound on the $A(\thetav)$.}

A nice property of the above approximation is that it leads to a simple coordinate descent algorithm. Fix all $\mu_j(x_j)$ except for $\mu_i(x_i)$ then one can show the following setting for $\mu_i(x_i)$ will maximize the objective \eqref{eq:naive_mean_field}:
\be
\mu_i(x_i) \propto \exp{\theta_i(x_i) + \sum_{j\in \nei{i}} \sum_{x_j} \theta_{ij}(x_i,x_j)\mu_j(x_j) } \label{eq:naive_iter}
\ee
%Iterating through \eqref{eq:naive_iter} will converge, since \eqref{eq:naive_mean_field} is a concave function.
%\atodo{what you wrote above (in comment) is wrong. it's not concave}
The above algorithm will converge to a local optimum of $A_{I}(\thetav)$, and often provides good results in practice.

%% The above example demonstrates two important features of the chosen dependency graph.
%% First, the entropy should be easy to calculate.
%% Second, the resulting approximated variational equation should be convex, allowing easy optimization.
%% The second feature does not always hold, for example when the dependency graph is a tree. 
%% The model entropy is the Bethe entropy (see \eqref{eq:bethe_entropy}), which is not a concave function. 
%% Hence, local maximum may occur and the resulted algorithm may not converge\footnote{This is not BP; to get BP, the optimization should be over the local polytope, not the tree polytope}.
%% This leads us to the next approach - different approximation to the marginal polytope and the entropy.
\subsubsection{Belief Propagation}
\label{sec:belief}
Belief-Propagation (BP) is one of the most widely used approximate inference algorithms. One of the reasons for this popularity is its excellent performance
on tasks such as decoding error correcting codes \cite{richardson2001design} and machine vision applications \cite{felzenszwalb2006efficient}.
 
Like mean field, BP is an iterative algorithm, which updates a set of variables at each iteration. However, whereas in mean field the variables are single 
node marginals, in BP they are messages between nodes. Specifically, BP involves
updating messages $m_{i \to j}(x_j)$ that are functions of $x_j$ and are {\em sent} from node $i$ to node $j$.
%The message reflects the source vertex belief on the destination vertex probability.

%In each cycle\footnote{The order of the messages can effect convergence rate, see for example  \cite{elidan2012residual}.} 
%a message is passed from a vertex to all its neighbors. It is a function of the factor connecting the two vertices and the messages from all neighbors except the destination neighbor.
%BP continues to cycle until the difference in the messages is very small, or after a fixed number of cycles.
There are many schedules for updating messages (sequentially, in parallel, or
%U via some
dynamically chosen as in  \cite{elidan2012residual}). In all of these,
the new value for $m_{i \to j}^{t}(x_j)$ is a function of all messages $m^{t-1}_{k\to i}(x_i)$ as follows:
\be
\label{eq:belief_propagation}
m_{i \to j}^{t}(x_j) \propto \sum_{x_i \in\cX} \exp{\theta_{i,j}(x_i,x_j)+\theta_{i}(x_i)}\prod_{k \in \nei{i} \setminus j } m_{k \to i}^{t-1} (x_i)
\ee 

The resulting approximation for the marginals are given by:
\bean
\tau_i(x_i) &\propto& \exp{\theta_i(x_i)} \prod_{k \in \nei{i}} m^T_{k \to i}(x_i) \label{eq:bp_single_marginal}\\
\tau_{ij}(x_i,x_j) &\propto& \exp{\theta_{ij}(x_i,x_j)+\theta_i(x_i)+\theta_j(x_j)} \prod_{k \in \nei{i}\setminus j} m_{k \to i}^{T} (x_i) \prod_{k \in \nei{j}\setminus i}m_{k \to j}^{T} (x_j)\label{eq:bp_pairwise_marginal}
\eean
Note that $\tau_i,\tau_{ij}$ need not correspond to marginals of any distribution $P(\xx)$. In other words, they may be outside the marginal polytope.
%\atodo{cite yedidia for an example of this - i did not see it at yedidia so i cite wainright} 
Thus, they are often referred to as ``pseudo-marginals'' \cite{wainwright2008graphical}, a term we will also use here.


The empirical success of BP has prompted researchers to better understand its theoretical properties, leading to a wide range of elegant results \cite{tatikonda2002loopy,wainwright2003tree,ihler05b}.
%\atodo{cite some}
We describe some of these below, with a focus on the variational perspective.

First, when the graph $G$ is a tree BP is exact, since it essentially implements
dynamic programming on the tree. There are other model families  where the error of BP marginals can be bounded (e.g., ``tree-like'' graphs with long cycles as in  \cite{dembo2010ising}. See also  \cite{heinemann2014inferning} in this thesis).
However, in the general case the quality of the BP approximation is unknown. 
Moreover, there are cases where BP does not converge, and even when it does, the resulting pseudo-marginals may depend on the initial messages.
 Despite these points, BP often yields  good results in practice  \cite{willsky2002multiresolution,kschischang2003codes,loeliger2004introduction}.

We next turn to a variational formulation of BP, as derived in the seminal work of  \citet{yedidia2000generalized, yedidia2003understanding}. Briefly, these results
show that the fixed points of BP are local optima of a certain variational approximation,\footnote{Later,  \cite{heskes2002stable} refined this result, and found that \textbf{stable} fixed point of BP are local minima of the variational approximation.} which starts
at \eqref{eq:variation_A} and approximates both the marginal polytope and entropy term.

% where the authors found that the fix points of BP are local minima of the Bethe free energy of the system\footnote{Later,  \cite{heskes2002stable} refined this result to \textbf{stable} fix point of BP are local minima of the Bethe approximation.}.
%This result not only gave meaning to BP fix points, it also allowed theoretical analysis of BP. 
%I will now present this result; prior to this, some definitions are necessary.

\red{
Consider an outer bound on the marginal polytope.
Note that any set of marginals in $\margpoly^G$ needs to be consistent, in the sense that $\mu_{ij}(x_i,x_j)$ needs to marginalize to $\mu_i(x_i)$.
This implies that requiring a set of pseudo-marginals to be pairwise consistent yield a set $\lclmargpoly$ such that $\lclmargpoly \supseteq \margpoly^G$.
The formal definition of $\lclmargpoly$ (also known as the {\em local marginal polytope}) is as follows:
%\atodo{Should add $G$ to $\lclmargpoly$, or remove it from $\margpoly^G$.} 
\be
\label{eq:local_polytope}
\lclmargpoly^G = \left\{\tauv \in \Re^d\left| 
\begin{array}{lr}
\forall i \in V & \sum_{x_i} \tau_i(x_i) = 1\\
\forall i \in V, \forall x_i \in \cX,\ \forall j \in \nei{j}& \sum_{x_j}\tau_{ij}(x_i,x_j) = \tau_i(x_i)\\
\forall i \in V,\ \forall ij \in E,\ x_i,x_j \in \cX &\tau_{ij}(x_i,x_j) \geq 0% \tau_i(x_i) \geq 0,\ 
\end{array}\right.\right\}
\ee 
When $G$ is a tree we in fact have $\margpoly^G = \lclmargpoly$.
}

\red{
%As mentioned earlier, the approximation of \eqref{eq:variation_A} includes two parts.
Next define the Bethe entropy, a function of given singleton and pairwise pseudo-marginals $\tauv \in \lclmargpoly$:
\bean
H_B(\tauv) &=& -\sum_{i} (1-d_i)\sum_{x_i}\tau_i(x_i)\log\tau_i(x_i) -\sum_{ij}\sum_{x_i,x_j}\tau_{ij}(x_i,x_j)\log\tau_{ij}(x_i,x_j)\label{eq:bethe_entropy}\\
&=&-\sum_{i}\sum_{x_i}\tau_i(x_i)\log\tau_i(x_i) -\sum_{ij}\sum_{x_i,x_j}\tau_{ij}(x_i,x_j)\log\frac{\tau_{ij}(x_i,x_j)}{\tau_i(x_i)\tau_j(x_j)} \label{eq:bethe_entorpy_information}
\eean
For a tree shaped MRF, with marginals $\tauv$, we in fact have $H_B(\tauv) = H(\thetav(\tauv))$ (see  \cite{yedidia2003understanding})
}


Using the above two definitions, we can now write the so called Bethe approximation to the partition function as:
\be
\label{eq:bethe_approximation}
A_B(\thetav) = \sup_{\tauv \in \lclmargpoly} \left\{\thetav \cdot \tauv + H_B(\tauv)\right\}
\ee
A related function used in the literature is the Bethe energy, which corresponds to $-A_B(\thetav)$.\\

The advantage of the above optimization is that the objective is simple to calculate, and the constraints are simple to check and express via
a polynomial number of inequalities. 

There are however two remaining difficulties. First, the maximizer of $\eqref{eq:bethe_approximation}$ is not necessarily in the marginal-polytope, and hence not a ``real'' marginal. Second, the resulting optimization problem is no longer convex, as the Bethe entropy is a non-convex function.
In fact,  as far as we know, there is no result analyzing the general hardness of the Bethe approximation problem.\footnote{In  \cite{weller2012bethe} a polynomial algorithm is proposed for approximating the Bethe free energy up to arbitrary precision in binary attractive models.}

We are now ready to state the result relating the BP algorithm to the Bethe variational approximation in \eqref{eq:bethe_approximation}.
%The claim by  \cite{yedidia2000generalized} can now be quoted:
\begin{claim}  \cite{yedidia2000generalized}
\label{thm:bp_bethe}
Let  $\mm$ be a set of messages as in \eqref{eq:belief_propagation}, and let $\tauv$ be the calculated pseudo-marginal as in \eqref{eq:bp_pairwise_marginal}.
Then the pseudo-marginals are a fix-point of BP, if and only if they are zero gradient points of the Bethe Free energy \eqref{eq:bethe_approximation}.
\end{claim}
%So the importance of \eqref{eq:bethe_approximation} to inference, is its connection to BP.

The proof is as follows. Writing the Lagrangian of this optimization problem,
\bea
\mathcal{L}(\thetav,\tauv,\lambdav) &=& -\thetav \cdot \tauv - H_B(\tauv) \\
&+& \sum_i \lambda_i \left(1-\sum_{x_i} \tau_i(x_i)\right) + \sum_{i} \sum_{j \in \nei{i}}\sum_{x_i}\lambda_{j \to i, x_i}\left(\tau_i(x_i)-\sum_{x_j} \tau_{ij}(x_i,x_j)\right)
\eea
Using \eqref{eq:bethe_entorpy_information} for the Bethe entropy and remembering that the derivative of $x\log\frac{x}{a}$ is $\log\frac{x}{a}+1$, comparing the derivative to zero gives,\footnote{And using the constraint $\sum_{x_i}\tau_{ij}(x_i,x_j) = \tau_j(x_j)$},
\bea
\log{\tau_i(x_i)} &=& \theta_i(x_i)+ \sum_{j \in \nei{i}} \lambda_{j \to i,x_i}-(d_i-1)+\lambda_i\\
\log{\tau_{ij}(x_i,x_j)} &=&  \theta_{ij}(x_i,x_j) - \lambda_{j \to i,x_i} -  \lambda_{i \to j,x_j} +\log \tau_{i}(x_i) +\log \tau_{j}(x_j) -1
%\frac{\partial \mathcal{L}(\thetav,\tauv,\lambdav)}{\partial \tau_i(x_i)} &=& \theta_i(x_i) - \log{\tau_i(x_i)}-1 - \sum_{j \in \nei{i}} \lambda_{i \to j,x_i}\\
%\frac{\partial \mathcal{L}(\thetav,\tauv,\lambdav)}{\partial \tau_{ij}(x_i,x_j)} &=& \theta_{ij}(x_i,x_j) + \log{\tau_{ij}(x_i,x_j)} + 1 + \lambda_{i \to j,x_i} + \lambda_{j \to i,x_j}
\eea
Taking the exponent and rearranging we have,\footnote{$\lambda_i$ are part of the normalization along with the constants.}
\bea
\tau_i(x_i) &\propto& \exp{\theta_i(x_i)}\prod_{j \in \nei{i}} \exp{\lambda_{j \to i,x_i}}\\
\tau_{ij}(x_i,x_j) &\propto&  \exp{\theta_{ij}(x_i,x_j)+\theta_i(x_i)+\theta_j(x_j)} \prod_{k \in \nei{i}\setminus j} \exp{\lambda_{k \to i,x_i}} \prod_{k \in \nei{j}\setminus i} \exp{\lambda_{k \to j,x_j}}
\eea
which is exactly as in \eqref{eq:bp_single_marginal} and \eqref{eq:bp_pairwise_marginal} when BP converges.
Hence, BP fixed points are local minima of the Bethe energy.

The result in \claimref{thm:bp_bethe} prompted much followup research. These followup works can be roughly divided into three categories:
Extensions of BP, convergence of BP and bounding the error of BP.
 We begin by stating some implications of  \claimref{thm:bp_bethe}, followed by a short review of each category.

In the general case,  BP may have more than one fixed point.
%This empirical fact now has theoretical reasoning.
\claimref{thm:bp_bethe} provides a theoretical characterization of these fixed points.
It states that the local optima of the Bethe free energy are fixed points of BP.
%\atodo{If you want to say  local minimum then cite Heskes, or leave like this - I deleted stable} 
In addition, the Bethe free energy is a non-convex function, therefore multiple minima may exist, which is in agreement
with the empirical observation of multiple fixed points.
%Taken together, this  that BP may have more than one fixed point.

Note that not all BP fixed points are equally good in terms of the optimization problem \eqref{eq:bethe_approximation}.
Give two fixed points, the one with the better objective value may be a solution to \eqref{eq:bethe_approximation} while the other may not.
This goes against the common practice of initializing the messages $\boldsymbol{m}^0$ to a uniform distribution and running BP only once.
Multiple restarts of BP with random initialization may improve the quality of BP results, at least when calculating the partition function.\footnote{How to use the resulting pseudo-marginals of BP when multiple maxima exists can be an interesting research direction.}

The first extension of \claimref{thm:bp_bethe} was the Generalized Belief Propagation algorithm  \cite{yedidia2000generalized}. 
Instead of maximizing the Bethe energy, it maximizes the Kikuchi free energy - an extension from only pairwise interactions to larger subsets. This 
indeed resulted in improved marginal approximation in many cases.


One of the drawbacks of the Bethe approximation is the fact that the objective \eqref{eq:bethe_approximation} is not concave.
Tree Re-Weighted Belief Propagation (TRW)  \cite{wainwright2003tree} proposes an alternative entropy approximation which is concave.
This is achieved by proper weighting of the sum over edges in \eqref{eq:bethe_entorpy_information}.
%Its name comes from the interpretation of these weights.
\ignore{
The model parameter can be decomposed: $\thetav=\sum_{\mathfrak{t}} t_{\mathfrak{t}}\thetav^{\mathfrak{t}}$ such that each $\thetav^{\mathfrak{t}}$ is a sub-model\footnote{$\theta^{\mathfrak{t}}_{ij}$ is either $0$ or $\theta_{ij}$, and $\theta^{\mathfrak{t}}_i =\theta_i$} the structure of which is a tree, and $t_{\mathfrak{t}}\geq 0$, $\sum_{\mathfrak{t}} t_{\mathfrak{t}} = 1$.
The probability on trees induces a probability for each edge to appear in any one of the trees;
these probabilities are weights of the information part.
}
Another nice property of TRW is that the resulting log partition approximation is an upper bound on the true partition function. 
%This can easily be seen by using Jensen inequality\footnote{Remember that the partition function is a convex function of $\thetav$.} $Z(\thetav) = Z\left(\sum_{\mathfrak{t}} t_{\mathfrak{t}}\thetav^{\mathfrak{t}}\right) \leq \sum_{\mathfrak{t}} t_{\mathfrak{t}} Z(\thetav^{\mathfrak{t}})$ which is exactly the TRW approximation to the partition function.
% Both algorithms suggest a variation of BP messages to fit the change in energy.
A unified view of entropy based approximation algorithms are found in \cite{meshi2009convexifying}.

Another line of inference algorithms attempts to optimize the Bethe free energy directly.
For the binary case,  \cite{welling2001belief} and later  \cite{shin2012complexity} present two similar algorithms, but the latter gives time complexity guarantees. A general algorithm was given by  \cite{yuille2002cccp}, which uses a convex concave procedure (CCCP, see  \cite{yuille2002concave}) to optimize the Bethe or Kikuchi energies.

One of the basic desiderata of any algorithm is convergence in finite time. Unfortunately, it is easy to construct cases where the
BP messages do not converge to any limit. This has prompted much research on understanding conditions for guaranteeing BP convergence.
%So the question remain, can convergence be guaranteed under some model's constrains.
The first work to do so was  \cite{tatikonda2002loopy}, where a bound on the pairwise interaction was given to guarantee convergence.
For proving the bound, the computation tree of BP was used.
The computation tree describes the running of BP where each iteration is a level in the tree.
The root value is the pseudo-marginal of a specific vertex at iteration $t$, its direct descendants are the neighbors that send it a messages from iteration $t-1$, and their values are the pseudo-marginals at that iteration.
It continues to go back, until arriving at the leaves which are the initial messages at iteration $0$.
Taking the number of iterations to infinity, if the marginals at iteration $t$ are independent of the initial messages at iteration $0$, BP will converge.
This method has roots in statistical physics, where it is related to the questions of uniqueness of the Gibbs measure, or independence of boundary conditions.\footnote{These notions are important to the understanding of our second paper  \cite{heinemann2014inferning}, where each marginal should be independent of nodes which are at distance $l$.}

\claimref{thm:bp_bethe} is used in  \cite{heskes2004uniqueness} for finding convergence guarantees.
Since insuring the convexity of Bethe free energy implies convergence of BP, they define a set of conditions that guarantee this convexity.
Another method to guarantee BP convergence is the ability to bound the speed at which the messages get closer to each other.
In other words, if the distance (under any distance measure) between any two vectors of messages at time $t+1$ $\mm^{t+1}$,$\hat{\mm}^{t+1}$ is smaller than the distance at time $t$ by a rate that is always strictly smaller than one, then convergence of BP can be guaranteed: $ K d(\mm^{t+1}, \hat{\mm}^{t+1}) \leq d(\mm^t, \hat{\mm}^t)$ where $0\leq K<1$ and $d(\xx,\yy)$ is some distance function.
In both papers  \cite{mooij2007sufficient} and  \cite{roosta2008convergence} this intuition is used to provide bounds for BP convergence.
Note that all the above methods do not find conditions for BP convergence, but rather that BP will have a single fixed point.
This fact will be important in \cite{heinemann2012cannot}, presented in this thesis.

The relation between the Bethe partition function \eqref{eq:bethe_approximation} and the true partition function in \eqref{eq:variation_A} is still an open question.
For example, the Bethe partition function neither upper bounds nor lower bounds $A(\thetav)$ in the general case, but such a bound might exist in specific cases.
One of the first results characterizing BP approximations is \cite{AlanNips2007}, where it was shown that if all the pseudo-marginals have the same orientation (all larger or smaller than $0.5$), then the Bethe partition function lower bounds the exact partition function.
Later \cite{RuozziNips2012} proved a stronger result, showing that if the pairwise interactions are log sub-modular, then the Bethe partition function lower bounds the exact partition function (the case of attractive binary Ising models is an instance of this class).
In their proof, they used the important work of  \cite{vontobel2013counting}, in which the Bethe entropy was given a combinatorial interpretation.

To conclude, the connection between BP and the Bethe variational formulation contributes to the understanding of BP and can result in practical improvements.
The biggest shortcoming of this connection is that it says nothing with respect to BP messages before convergence. Specifically, when BP does not converge,
 the resulting pairwise beliefs are not in the local polytope and thus cannot be directly related to the Bethe free energy. Hence, the Bethe interpretation applies only
 at convergence. This has prompted researchers to develop algorithms that directly minimize the Bethe free energy. See \cite{welling2001belief,yuille2002cccp}.
%%U
%\red{
%So, if BP does not converge, the result is not in the local polytope, and hence cannot be related to the Bethe free energy.
%In other words, BP does not minimize the Bethe free energy, but rather seeks after another objective, one that when reached, minimizes the Bethe free energy.}
%%
Finally, we mention another interpretation as a re-parametrization algorithm. In this view, BP changes the parametrization of the MRF parameters $\thetav$ in order to optimize a bound on the partition function. For more details please refer to  \citet{wainwright2001tree,wainwright2002stochastic}.
\ignore{
\be
\mu_k(x_k;\thetav) = \frac{1}{Z(\thetav)}\sum_{\substack{\xx \\
s.t.\  \xx_k=x_k}}e^{\theta_k(x_k) + \sum_{j \in \nei{k}}\theta_{k,j}(x_k,x_j)}e^{\sum_{i \in V \setminus k}\theta_{i}(x_i) +\sum_{\substack{ij \in E\\
 s.t.\  i,j \ne k}}\theta_{ij}(x_i,x_j)}
%} {\sum_{\hat{x}_k}e^{\theta_k(\hat{x}_k) + \sum_{j \in \nei{k}}\theta_{k,j}(\hat{x}_k,x_j)}e^{\sum_{i \in V \setminus k}\theta_{i}(x_i) +\sum_{\substack{ij \in E\\
% s.t.\  i,j \ne k}}\theta_{ij}(x_i,x_j)}}
\ee
Denote by $\thetav^{\setminus k}$ the model where we remove all factors involve the vertex $k$.
Now the marginal of the neighbors of $k$ in  that model is
\be
\muv_{\nei{k}}(\xx_{\nei{k}}; \thetav^{\setminus k}) \approx \sum_{\substack{\hat{\xx}\\
s.t. \hat{\xx}_{\nei{k}} = \xx_{\nei{k}}}}  e^{\sum_{i \in V \setminus k}\theta_{i}(\hat{x}_i) +\sum_{\substack{ij \in E\\
 s.t.\  i,j \ne k}}\theta_{ij}(\hat{x}_i,\hat{x}_j)}
\ee
 With this we can write
\bea
\mu_k(x_k;\thetav)  &\approx& \sum_{\xx_{\nei{k}}} e^{\theta_k(x_k) + \sum_{j \in \nei{k}}\theta_{k,j}(x_k,x_j)} \muv_{\nei{k}}(\xx_{\nei{k}}; \thetav^{\setminus k})\\
 &\approx& e^{\theta_k(x_k)}  \prod_{j \in \nei{k}} \sum_{ x_j } e^{\theta_{k,j}(x_k,x_j)} \muv_{j}(x_j; \thetav^{\setminus k})\\
\eea
}

\red{
Next we will present a different approach for inference - sampling.
}


\subsection{Sampling}
\label{sec:sampling}
At first glance, sampling seems like the perfect solution for finding the marginals.
The marginals may be estimated as the mean of bounded random variables.
Using the Hoeffding bound, the probability of mistake greater than $\epsilon$ can be bound by $P(|p(x_i) - \frac{\sum_{m=1}^M \deltaF{\xx_i^m=x_i}}{M}| > \epsilon) < 2\exp{-\frac{2\epsilon^2M}{|\cX|^2}}$; meaning, the time complexity is $O(\frac{\log(\delta) |\cX|^2}{\epsilon^2})$\footnote{With a probability greater than $1-\delta$, the estimation error will be less than $\epsilon$.}.
The problem in this is, that sampling from a model as in \eqref{eq:ciluqe_prob} is hard in itself\footnote{This can be easily seen from redction to q-coloring, for example \cite{levin2009markov, bordewich2016mixing}}.
How, then, can a model be sampled?

Markov Chain Monte Carlo (MCMC) is a method of sampling from a distribution $P(\xx)$ while being able to sample only from distribution $Q(\xx)$.
In this, a Markov chain with two main features is constructed: first, it must be easy to move from one state to another; secondly, the stationary distribution of the Markov chain is $P(\xx)$.

To define a Markov chain, one needs to define the transition matrix between any two states.
The matrix rows should sum to one $\sum_j A_{i,j} = 1$, the coordinate $A_{i,j}$ defining the probability to move from state $i$ to state $j$.
In MCMC, this matrix of dimensions is $A \in [0,1]^{|\cX|^{p},|\cX|^{p}}$, meaning that each state is a full assignment of the model variables. 
The stationary distribution $\vv$ is a distribution over the states (over different assignments) that satisfies $A\vv=\vv$.
In MCMC, this distribution should be $P(\xx)$ - the model distribution.

Metropolis chains is an example of how the Markov chain can be constructed.
The transition matrix is defined as follows:
\be
A(\xx,\tilde{\xx}) = \left\{
\begin{array}{lr}
Q(\tilde{\xx}|\xx)\min\{1,\frac{Q(\xx|\tilde{\xx})P(\tilde{\xx})}{Q(\tilde{\xx}|\xx)P(\xx)}\} & \tilde{\xx} \neq \xx\\
1 - \sum_{\hat{\xx} \neq \xx} Q(\hat{\xx}|\xx)\min\{1,\frac{Q(\xx|\hat{\xx})P(\hat{\xx})}{Q(\hat{\xx}|\xx)P(\xx)}\} & \text{else}
\end{array} \right.
\ee
Note that $Q(\xx|\tilde{\xx})$ must only be irreducible\footnote{Irreducible defined: for all $\xx,\xx_0 \in \cX^p$ exists $k \in \naturalNumbers$ and $\xx^1, \ldots, \xx^k=\xx$, such that $\prod_{i=1}^k Q(\xx_i|\xx_{i-1}) >0$} - it does not necessarily have to be symmetric; hence, it could be the distribution over a tree or any easy to sample distribution.
Moreover, we need only to compute $\frac{P(\xx)}{P(\tilde{\xx})} $, hence the partition function is canceled, and the quantity is easy to calculate.

Glauber chains, or Gibbs sampler, is another method to define the Markov transition matrix.
The idea is simple, in each iteration only a small number of variables will change, this group denoted by $\cI \subseteq [1,\ldots,p]$, $|\cI| = k$.
Then, the possible next states are the ones that differ from the current state $\xx$ only in the indices in $\cI$, this set denoted by $\mathcal{D}(\xx,\cI) = \left\{\tilde{\xx} \in \cX^p | \forall i \notin \cI,\ \xx_i = \tilde{\xx}_i \right\}$.
The set $\mathcal{D}^k(\xx)= \cup_{\cI, |\cI|=k} \mathcal{D}(\xx,\cI)$ denotes the set of all $\tilde{\xx}$ that differ from $\xx$ only in $k$ entries.
Now the transition matrix is defined as,
\be
A(\xx,\tilde{\xx}) = \left\{
\begin{array}{lr}
0 & \tilde{\xx} \notin \mathcal{D}^k(\xx)\\
\frac{(p-k)!k!}{p!}\frac{P(\tilde{\xx})}{\sum_{\hat{\xx} \in \mathcal{D}(\xx,\cI)}P(\hat{\xx})} & \tilde{\xx} \in \mathcal{D}^k(\xx), \xx \neq \tilde{\xx}\\
\frac{(p-k)!k!}{p!}\sum_{\cI, |\cI|=k}\frac{P(\xx)}{\sum_{\hat{\xx} \in \mathcal{D}(\xx,\cI)}P(\hat{\xx})}& \xx = \tilde{\xx}
\end{array} \right.
\ee
Note that as in the case above, only the ratio of $P(\xx)$ needs to be computed, hence the partition function can be canceled.

The remaining question is, how fast are we guaranteed to sample from the stationary distribution when starting from an arbitrary assignment $\xx$.
This constant is called the \textit{mixing time}.
The mixing time is defined as the maximum number of moves (transitions between states) necessary until the visited states distribution is $\epsilon$ close to the stationary distribution while starting from any state.
\be
\tau_{\epsilon}  = \min_{\tau \in \naturalNumbers}\sup_{\vv \in \Omega} \left\{\tau | d\left(A^{\tau}\vv,P(\xx)\right) < \epsilon\right\}
\ee
where $\Omega = \{ \vv \in [0,1]^{|\cX|^{p}} | \sum_i \vv_i = 1\}$, and $d(\vv,\tilde{\vv})$ is any distance function between probabilities.\footnote{ From the above condition it is easy to see the connection to the spectral gap (the distance between the largest to the second eigenvalue) and the mixing time. For more information see\cite{levin2009markov}}  .
The mixing time depends on the model features, such as interaction strength, dependency structure etc.
Hence, models with a small mixing time can be easily sampled, and inferring the marginals is doable.
This fact will be important when talking about inferning \secref{sec:inferning}.

The next section covers learning graphical models.


