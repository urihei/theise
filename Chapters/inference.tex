% !TEX root =../main.tex
\subsection{Approximate inference}
\label{sec:approx}
In order to use graphical models in practice, one must solve the inference tasks
described in Section \ref{sec:inference}. As mentioned earlier, in the general case
this is an ``intractable'' task (by which we mean it has no known polynomial time algorithm). This has prompted much exciting research on approximate inference algorithm, combining
tools from combinatorial optimization, convex optimization and graph theory. 

Below, we will provide a brief overview of approximate inference approaches, and describe two of the most widespread paradigms: variational methods and sampling based methods.  
We will focus only on the problem of finding the marginals, since it is more relevant for 
the learning problem, which is the focus of this thesis.

\subsection{Variational Methods}
\label{sec:variational_methods}
Variational methods are a large class of approximate inference algorithms. The key
underlying approach in all of these is casting inference as a continuous (as opposed to discrete) optimization problem. Our review below largely follows the excellent and rigorous introduction to the topic in \cite{wainwright2008graphical}. 

The goal of marginal inference is to calculate the marginals of the model $P(\xx;\thetav)$. Here we focus on calculating singleton marginals $\mu_i^{\thetav}(x_i)$ and pairwise marginals  $\mu^{\thetav}_{ij}(x_i, x_j)$ (see \eqref{eq:pratition_derivative}). In what follows we will denote by $\muv$ the vector constructed from concatenating all these marginals, and denote its dimension by $d$.\footnote{For example, in the case of binary variables we will have $d = 2|V| + 4|E|$.} 

The variational approach proceeds by defining some function $g(\muv)$ of the marginals, and maximizing this function, subject to some constraints, to obtain the approximate marginals. A first step to realizing such an approach is to understand which values of $\muv$ are possible. In other words, which vectors $\muv$ are marginals of some model $P(\xx;\thetav)$. Clearly we need not consider vectors $\muv$ that cannot be marginals of any model.

The above point leads to the definition of the ``Marginal Polytope''  \cite{wainwright2008graphical}. The Marginal Polytope, which we denote by $\margpoly^G$,  is a set of $\muv$ vector, which are the marginals of
``some'' distribution. The definition does not require this distribution to be an MRF. However, as shown in  \cite{wainwright2008graphical}, any $\muv$ in the interior of $\margpoly^G$ is in fact also the marginals of an MRF $P(\xx;\thetav)$, for some $\thetav$. The formal definition of $\margpoly^G$ is:\atodo{The definition here should not include $\thetav$ but instead a general distribution on $\xx$} 
\be
\label{eq:margpoly}
\margpoly^G = \left\{ \muv \in [0,1]^d\ \left| 
\begin{array}{lr}
  \exists \thetav \in \Omega^G\ s.t. \\
  \forall i \in V \land \forall x_i \in \cX &   P(x_i;\thetav) = \mu_i(x_i)\\
  \forall ij \in E \land \forall x_i, x_j \in \cX &P(x_i,x_j;\thetav) = \mu_{ij}(x_i,x_j)
\end{array} \right. \right\}
\ee

Next, we define a continuous optimization problem whose maximizer will be $\muv^{\thetav}$, and whose maximum value will be the log partition function $A(\thetav)$ (see \eqref{eq:partition_function}). 

Since $A(\thetav)$ is a convex function, it has a conjugate convex function $A^*(\muv)$, defined as follows \atodo{Add citation to Boyd convex optimization book}: \atodo{Including $\Omega$ here is also confusing, so best to drop it everywhere}
\be
\label{eq:conjugate_partition}
A^*(\muv) = \sup_{\thetav \in \Omega^G} \left\{\muv \cdot \thetav - A(\thetav)\right\}
\ee
Since the conjugate of $A^*(\muv)$ is $A(\thetav)$, we can cast $A(\thetav)$ as the following optimization problem:
\be
A(\thetav) = (A^{**}(\thetav) = \sup_{\muv} \left\{\muv \cdot \thetav - A^*(\muv)\right\}
\label{eq:conjconj}
\ee
To further simplify this problem, we note several properties of $A^*(\muv)$. First
if $\muv\notin\margpoly^G$ then $A^*(\muv) = \infty$. Thus, we can limit optimization to
$\muv\in\margpoly^G$. Next, for any $\muv$ there exists a set of MRF parameters $\thetav(\muv)$ such that the marginals of $P(\xx;\thetav(\muv))$ are $\muv$.\footnote{The function $\thetav(\muv)$ is sometimes referred to as the mapping from mean parameters to canonical parameters. See \cite{wainwright2008graphical} for a thorough discussion} Denote the entropy of a distribution $P(\xx)$ by $H(P(\xx))$. Then for $\muv\in\margpoly^G$ the function $A^*(\muv)$ turns out to be $H(P(\xx;\thetav(\muv)))$.  

The above observations imply that \eqref{eq:conjconj} can be written as:
%Two points should be noted. First, if $\muv$ is in the interior of $\margpoly^G$ \footnote{This applies only for marginals in the inner part of the marginal polytope. For marginals on the boundaries, more delicate treatment should be given} then $A^*(\muv) = -H(P(\xx;\thetav(\muv)))$ the conjugate of the log partition function equals to the entropy of the probability the marginals of are $\muv$.
%Second, \eqref{eq:conjugate_partition} is similar to the maximum likelihood objective (see \secref{sec:max_likelihood}).
%It can be shown that $A(\thetav)$ is a convex function of $\thetav$, which leads to ${A^{*}}^* = A$. Having that, the variational expression of $A(\thetav)$ may be written as, 
\be
A(\thetav) = \sup_{\muv \in \margpoly^G}\left \{ \muv \cdot \thetav + H(P(\xx;\thetav(\muv))) \right\} \label{eq:variation_A} 
\ee
Also, since the gradient of $A(\thetav)$ is $\muv^{\thetav}$, it follows that:
\be
\muv^{\thetav}= \arg \sup_{\muv \in \margpoly^G}\left \{ \muv \cdot \thetav + H(P(\xx;\thetav(\muv))) \right\} \label{eq:arg_variation_A}
\ee
\eqref{eq:arg_variation_A} and \eqref{eq:variation_A} offer an optimization based view of marginals and partition function calculation. They form the starting point for most
variational approximations.


Not surprisingly, solving the above optimization problem is hard, for two reasons. First,  optimizing over the marginal-polytope $\margpoly^G$ is hard in general (even if the objective is linear). Describing $\margpoly^G$ requires a number of inequalities that is exponential in $d$.  Second, both calculating $\muv^{\thetav}$ is hard, as is calculating the entropy of the resulting MRF.

The advantage of the above representation lies in the simplicity of approximating its constraints and objectives, to obtain a more ``managable'' optimization problem.\footnote{We use the loose term managable, since not all approximations correspond to polynomial time solvable optimization problem. For example mean field attempts to solve a non-convex problem and may obtain sub-optimal solutions}  
%% Two methods of approximation will next be presented. The first is the mean-field method \cite{peterson1987mean} where both the marginal-polytope and the entropy are restricted to a sub-graph of $G$ such that both are easy to calculate.
%% The second method gives different approximations to the entropy and to the marginal polytope.
%% For example, the local polytope (defined in \eqref{eq:local_polytope}) will approximate the marginal-polytope, while the entropy is approximated by the Bethe-entropy. 
%% The resulting approximation is the Bethe approximation, which is related to the known algorithm Belief-Propagation\cite{pearl1986fusion, yedidia2000generalized}.
\subsubsection{Mean Field}
Mean field methods have a long history, starting from statistical physics, and later popularized in the machine learning community.\atodo{citation from physics. Can also add \cite{peterson1987mean}} There are various ways of deriving them, and here we will provide one based on the variataional view of \eqref{eq:variation_A}.\footnote{A popular and earlier derivation, casts mean field as finding a simple distribution that approximates $P(\xx;\thetav)$ in a KL divergence sense.} 
%The term mean field methods (e.g., see \cite{peterson1987mean}) corresponds to approximate inference approaches that that approximate a complex model by a simpler one.
%In inference, it approximates the real model marginals by marginals of a simpler model.

%\atodo{It's incorrect ot say mean field uses $\margpoly^{G_I}$ since that contains only singleton marginals}
We begin by considering all distributions $P(\xx)$ corresponding to a set of $\nvars$ independent variables, as in \eqref{eq:independent}. The pairwise marginals of such distributions will also indepednent, namely given by $\mu_{ij}(x_i,x_j) = \mu_i(x_i)\mu_j(x_j)$. Such distribtions are therefore always in $\margpoly^G$. This implies that the following set $\margpoly^I$ is a subset of $\margpoly^G$ (namely $\margpoly^{I} \subseteq \margpoly^{G}$):\atodo{Strictly speaking this is not a marginal polytope...}
%More specifically, it replaces the marginal polytope (of the true dependencies graph) and the true entropy, with the marginal polytope and entropy of a simpler dependencies graph.
%I will illustrate this technique by choosing the simplest graph - the independent graph denoted by $G_I$.
%In this case, the marginal polytope sums down to,
\be
\label{eq:margpoly_independent}
\margpoly^{I} = \left\{\muv \in \Re^{d}\left|
\begin{array}{lr}
\forall i \in V,\ \forall x_i \in \cX & \mu_i(x_i) \geq 0 \\
\forall i \in V & \sum_{x \in \cX} \mu_i(x_i) = 1\\
\forall ij \in E,\ x_i,x_j \in \cX & \mu_{ij}(x_i,x_j) = \mu_i(x_i)\mu_j(x_j)
\end{array}
\right.\right\}
\ee
%Note that the dimension of the marginals is with respect to the original graph,
%moreover, $\margpoly^{I} \subseteq \margpoly^{G}$\footnote{Note that marginals corresponding to the pairwise interaction are constrained to reflect independence, a constraint that does not exist in the original marginal polytope.}.\\

We next turn to calculating $A^*(\muv)$ for $\muv\in\margpoly^{I}$. It is easy to see that for any $\muv\\margpoly^{I}$  the distribution $P(\xx) = \prod_{i} \mu_i(x_i)$ has pairwse marginals $\mu_{ij}(x_i,x_j) =  \mu_i(x_i)\mu_j(x_j)$. Since $P$ is an MRF, it follows that
$\thetav(\muv)$ are the parameters specifying $P$. Also, the entropy of $P$ is just the sum of individual entropies:
\be
H_{I}(\muv) = -\sum_{i}\sum_{x_i} \mu_i(x_i)\log\mu_i(x_i)
\ee 

Putting all of the above together we have the mean field approximation of the partition function:  
\be
A_{I}(\thetav) = \sup_{\muv \in \margpoly^{I}}\left \{ \muv \cdot \thetav + H_{I}(\muv))) \right\} \label{eq:naive_mean_field}~,
\ee
Note that $A(\thetav) \geq A_{I}(\thetav)$ since by restricting the marginal polytope the result can only decrease, and $A^*(\muv)$ is exact.

A nice property of the above approximation is that it leads to a simple coordinate descent algorithm. Fix all $\mu_j(x_j)$ except for $\mu_i(x_i)$ then one can show the following setting for $\mu_i(x_i)$ will maximize the objective \eqref{eq:naive_mean_field}:
\be
\mu_i(x_i) \propto \exp{\theta_i(x_i) + \sum_{j\in \nei{i}} \sum_{x_j} \theta_{ij}(x_i,x_j)\mu_j(x_j) } \label{eq:naive_iter}
\ee
%Iterating through \eqref{eq:naive_iter} will converge, since \eqref{eq:naive_mean_field} is a concave function.
\atodo{what you wrote above (in comment) is wrong. it's not concave}
The above algorithm will converge to a local optimum of $A_{I}(\thetav)$, and often provides good results in practice.

%% The above example demonstrates two important features of the chosen dependency graph.
%% First, the entropy should be easy to calculate.
%% Second, the resulting approximated variational equation should be convex, allowing easy optimization.
%% The second feature does not always hold, for example when the dependency graph is a tree. 
%% The model entropy is the Bethe entropy (see \eqref{eq:bethe_entropy}), which is not a concave function. 
%% Hence, local maximum may occur and the resulted algorithm may not converge\footnote{This is not BP; to get BP, the optimization should be over the local polytope, not the tree polytope}.
%% This leads us to the next approach - different approximation to the marginal polytope and the entropy.

\subsubsection{Belief Propagation}
\label{sec:belief}
Belief-Propagation (BP) is a message-passing algorithm.
In each cycle\footnote{The order of the messages can effect convergence rate, see for example \cite{elidan2012residual}.} a message is passed from a vertex to all its neighbors.
The message reflects the source vertex belief on the destination vertex probability.
It is a function of the factor connecting the two vertices and the messages from all neighbors except the destination neighbor.
BP continues to cycle until the difference in the messages is very small, or after a fixed number of cycles.
The messages may be written as, 
\be
\label{eq:belief_propagation}
m_{i \to j}^{t}(x_j) \propto \sum_{x_i \in\cX} \exp{\theta_{i,j}(x_i,x_j)+\theta_{i}(x_i)}\prod_{k \in \nei{i} \setminus j } m_{k \to i}^{t-1} (x_i)
\ee 
while resulted pseudo-marginals are calculated by,
\bean
\tau_i(x_i) &\propto& \exp{\theta_i(x_i)} \prod_{k \in \nei{i}} m^T_{k \to i}(x_i) \label{eq:bp_single_marginal}\\
\tau_{ij}(x_i,x_j) &\propto& \exp{\theta_{ij}(x_i,x_j)+\theta_i(x_i)+\theta_j(x_j)} \prod_{k \in \nei{i}\setminus j} m_{k \to i}^{T} (x_i) \prod_{k \in \nei{j}\setminus i}m_{k \to j}^{T} (x_j)\label{eq:bp_pairwise_marginal}
\eean

BP is exact when the dependency graph is a tree - the pseudo-marginals are the exact marginals $\tauv^{\thetav} = \muv^{\thetav}$.
There are other model families  where the error of BP marginals can be bound\footnote{Tree-like model is an example of such a family\cite{dembo2010ising}; this fact is utilized in my second paper\cite{heinemann2014inferning}.}. 
However, in the general case the quality of the pseudo-marginals is unknown. 
Moreover, there are cases where BP does not converge, and even when it does, the resulted pseudo-marginals may depend on the initial messages $\boldsymbol{m}^0$.
Despite these points, BP gives good results in practice \cite{willsky2002multiresolution,loeliger2004introduction,kschischang2003codes}.

The first step in understanding BP is to give meaning to the resulted pseudo-marginals.
This was acheived by \cite{yedidia2000generalized, yedidia2003understanding}, where the authors found that the fix points of BP are local minima of the Bethe free energy of the system\footnote{Later, \cite{heskes2002stable} refined this result to \textbf{stable} fix point of BP are local minima of the Bethe approximation.}.
This result not only gave meaning to BP fix points, it also allowed theoretical analysis of BP. 
I will now present this result; prior to this, some definitions are necessary.

As mentioned earlier, the approximation of \eqref{eq:variation_A} includes two parts.
First, let us define the Bethe entropy - an approximation of the true entropy,
\bean
H_B(\tauv) &=& -\sum_{i} (1-d_i)\sum_{x_i}\tau_i(x_i)\log\tau_i(x_i) -\sum_{ij}\sum_{x_i,x_j}\tau_{ij}(x_i,x_j)\log\tau_{ij}(x_i,x_j)\label{eq:bethe_entropy}\\
&=&-\sum_{i}\sum_{x_i}\tau_i(x_i)\log\tau_i(x_i) -\sum_{ij}\sum_{x_i,x_j}\tau_{ij}(x_i,x_j)\log\frac{\tau_{ij}(x_i,x_j)}{\tau_i(x_i)\tau_j(x_j)} \label{eq:bethe_entorpy_information}
\eean
Second, define the approximation of the marginal-polytope with the local-polytope.
\be
\label{eq:local_polytope}
\lclmargpoly = \left\{\tauv \in \Re^d\left| 
\begin{array}{lr}
\forall i \in V & \sum_{x_i} \tau_i(x_i) = 1\\
\forall i \in V, \forall x_i \in \cX,\ \forall j \in \nei{j}& \sum_{x_j}\tau_{ij}(x_i,x_j) = \tau_i(x_i)\\
\forall i \in V,\ \forall ij \in E,\ x_i,x_j \in \cX &\tau_{ij}(x_i,x_j) \geq 0% \tau_i(x_i) \geq 0,\ 
\end{array}\right.\right\}
\ee 
The Bethe approximation to the partition function can now be written.
\be
\label{eq:bethe_approximation}
A_B(\thetav) = \sup_{\tauv \in \lclmargpoly} \left\{\thetav \cdot \tauv + H_B(\tauv)\right\}
\ee
and the Bethe energy is $-A_B(\thetav)$.\\
In contrast to the variational free energy, the Bethe free energy is simple to calculate, since the local polytope is described by a linear number of constraints and the Bethe entropy calculation is straightforward .
This, however, comes with a price.
First, the result does not necessarily belong to the marginal-polytope hence not a ``real'' marginal.
Second, the resulted optimization is no longer convex, as the Bethe entropy is a non-convex function while the true one is convex.
Hence, the time complexity of optimizing this objective is unknown - I am not aware of any result pertaining to the time complexity of Bethe approximation\footnote{In \cite{weller2012bethe} a polynomial algorithm is suggested that approximates the Bethe free energy up to arbitrary precision under some models constraints.}.

The claim by \cite{yedidia2000generalized} can now be quoted:
\begin{claim}
\label{thm:bp_bethe}
Let  $\mm$ be a set of messages as in \eqref{eq:belief_propagation}, and let $\tauv$ be the calculated pseudo-marginal as in \eqref{eq:bp_pairwise_marginal}.
Then the pseudo-marginals are a fix-point of BP, if and only if they are zero gradient points of the Bethe Free energy \eqref{eq:bethe_approximation}.
\end{claim}
%So the importance of \eqref{eq:bethe_approximation} to inference, is its connection to BP.

The proof is as follows. Writing the Lagrangian of this optimization,
\bea
\mathcal{L}(\thetav,\tauv,\lambdav) &=& -\thetav \cdot \tauv - H_B(\tauv) \\
&+& \sum_i \lambda_i \left(1-\sum_{x_i} \tau_i(x_i)\right) + \sum_{i} \sum_{j \in \nei{i}}\sum_{x_i}\lambda_{j \to i, x_i}\left(\tau_i(x_i)-\sum_{x_j} \tau_{ij}(x_i,x_j)\right)
\eea
Using \eqref{eq:bethe_entorpy_information} for the Bethe entropy and remembering the derivative of $x\log\frac{x}{a}$ is $\log\frac{x}{a}+1$, the derivatives compare to zero,  giving\footnote{And using the constraint $\sum_{x_i}\tau_{ij}(x_i,x_j) = \tau_j(x_j)$},
\bea
\log{\tau_i(x_i)} &=& \theta_i(x_i)+ \sum_{j \in \nei{i}} \lambda_{j \to i,x_i}-(d_i-1)+\lambda_i\\
\log{\tau_{ij}(x_i,x_j)} &=&  \theta_{ij}(x_i,x_j) - \lambda_{j \to i,x_i} -  \lambda_{i \to j,x_j} +\log \tau_{i}(x_i) +\log \tau_{j}(x_j) -1
%\frac{\partial \mathcal{L}(\thetav,\tauv,\lambdav)}{\partial \tau_i(x_i)} &=& \theta_i(x_i) - \log{\tau_i(x_i)}-1 - \sum_{j \in \nei{i}} \lambda_{i \to j,x_i}\\
%\frac{\partial \mathcal{L}(\thetav,\tauv,\lambdav)}{\partial \tau_{ij}(x_i,x_j)} &=& \theta_{ij}(x_i,x_j) + \log{\tau_{ij}(x_i,x_j)} + 1 + \lambda_{i \to j,x_i} + \lambda_{j \to i,x_j}
\eea
taking the exponent and rearranging\footnote{$\lambda_i$ are part of the normalization along with the constants.} we have,
\bea
\tau_i(x_i) &\propto& \exp{\theta_i(x_i)}\prod_{j \in \nei{i}} \exp{\lambda_{j \to i,x_i}}\\
\tau_{ij}(x_i,x_j) &\propto&  \exp{\theta_{ij}(x_i,x_j)+\theta_i(x_i)+\theta_j(x_j)} \prod_{k \in \nei{i}\setminus j} \exp{\lambda_{k \to i,x_i}} \prod_{k \in \nei{j}\setminus i} \exp{\lambda_{k \to j,x_j}}
\eea
which is exactly as \eqref{eq:bp_single_marginal} and \eqref{eq:bp_pairwise_marginal} when BP converges.
Hence, BP fix points are local minima of the Bethe energy.

\claimref{thm:bp_bethe} promotes a wide research.
This research can be divided to three categories:
Extensions of BP, convergence of BP and bounding the error of BP.
Some of the direct implications of this claim will now be listed, after which a short review of each category will follow.

As was written before, BP may have more than one fix point.
This empirical fact now has theoretical reasoning. 
\claimref{thm:bp_bethe} states that local minima of the Bethe free energy is a stable fix point of BP.
In addition, the Bethe free energy is a non-convex function, therefore multiple minima may exist.
Taken together, this brings to the result that BP may have more than one fix point.
Note that not all fix points have the same quality, at least regarding the solution of \eqref{eq:bethe_approximation}.
Two fix points where one has a lower Bethe free energy should not be treated the same - the one with the lower energy could be a solution to \eqref{eq:bethe_approximation} while the other could not.
This suggests against the common practice of initializing the messages $\boldsymbol{m}^0$ to uniform distribution and running BP only once.
Multiple running of BP with random initialization may improve the quality of BP results, at least when calculating the partition function)\footnote{How to use the resulted pseudo-marginals of BP when multiple maximum exist can be an interesting research direction.}.

The most immediate extension to BP using \claimref{thm:bp_bethe}, was the Generalized Belief Propagation \cite{yedidia2000generalized}. 
Instead of maximizing the Bethe energy, it maximizes the Kikuchi free energy - an extension from only pairwise interaction to larger areas.
One of the drawbacks of the Bethe approximation is the fact that the Bethe entropy is not a concave function.
Tree Re-Weighted Belief Propagation (TRW) \cite{wainwright2003tree} suggests a surrogate entropy which is concave.
This is acheived by giving weights to the information part (the second summation) in \eqref{eq:bethe_entorpy_information}.
Its name comes from the interpretation of these weights.
The model parameter can be decomposed: $\thetav=\sum_{\mathfrak{t}} t_{\mathfrak{t}}\thetav^{\mathfrak{t}}$ such that each $\thetav^{\mathfrak{t}}$ is a sub-model\footnote{$\theta^{\mathfrak{t}}_{ij}$ is either $0$ or $\theta_{ij}$, and $\theta^{\mathfrak{t}}_i =\theta_i$} the structure of which is a tree, and $t_{\mathfrak{t}}\geq 0$, $\sum_{\mathfrak{t}} t_{\mathfrak{t}} = 1$.
The probability on trees induces a probability for each edge to appear in any one of the trees;
these probabilities are weights of the information part.
This gives another advantage to TRW - the resulted log partition function is an upper bound to the true partition function. 
This can easily be seen by using Jensen inequality\footnote{Remember that the partition function is a convex function of $\thetav$.} $Z(\thetav) = Z\left(\sum_{\mathfrak{t}} t_{\mathfrak{t}}\thetav^{\mathfrak{t}}\right) \leq \sum_{\mathfrak{t}} t_{\mathfrak{t}} Z(\thetav^{\mathfrak{t}})$ which is exactly the TRW approximation to the partition function.
Both algorithms suggest a variation of BP messages to fit the change in energy.
A unified view of these two algorithms (and others) may be found in \cite{meshi2009convexifying}.

Another line of inference algorithms attempts to optimize the Bethe free energy directly.
For the binary case, \cite{welling2001belief} and later \cite{shin2012complexity} give a similar algorithm, but the latter gives time complexity guarantees.
A general algorithm was given by \cite{yuille2002cccp}, which uses CCP\cite{yuille2002concave} to optimize the Bethe or Kikuchi energies.

One of the basic requirements of any algorithm is to be certain that after some time it will return a result.
In case of BP, it means that the messages will converege.
Since a model that BP does not converge on is easy to construct, for BP this translates into finding model constraints that guarantee BP convergence. 
%So the question remain, can convergence be guaranteed under some model's constrains.
The first work to do so was \cite{tatikonda2002loopy}, where a bound on the pairwise interaction was given to insure convergence.
For proving the bound, the computation tree of BP was used.
The computation tree describes the running of BP where each time is a level in the tree.
The root value is the pseudo-marginal of a specific vertex at time $t$ , its direct descendents are the neighbors that send it a message $t-1$, and their values are the pseudo-marginal at that time.
It continues to go back in time, until arriving at the leaves which are the initial messages at time $0$.
Taking the time to infinity, if the marginals in time $t$ are independent of the initial messages at time $0$, BP will converge.
This method comes from the physics world, where it is called uniqueness of Gibbs measure, or independence of boundary conditions\footnote{These notions are important to the understanding of our second paper \cite{heinemann2014inferning}, where each marginal should be independent of nodes which are at distance $l$.}.

\claimref{thm:bp_bethe} is used in \cite{heskes2004uniqueness} for finding convergence guarantees.
Since insuring the convexity of Bethe free energy implies convergence of BP, they define a set of conditions that guarantee this convexity.
Another method to guarantee BP convergence, is the ability to bound the speed at which the messages get closer to each other.
In other words, if the distance (in any distance measure) between any two vectors of messages at time $t+1$ is smaller than the distance at time $t$ by a rate that is always strictly smaller than one, then convergence of BP can be guaranteed $ K d(\mm^{t+1}, \hat{\mm}^{t+1}) \leq d(\mm^t, \hat{\mm}^t)$ where $0\leq K<1$ and $d(\xx,\yy)$ is some distance function.
In both papers \cite{mooij2007sufficient} and \cite{roosta2008convergence} this intuition is used to provide bounds for BP convergence.
Note that all the listed methods do not find conditions to BP convergence, but rather that BP will have a single fix point.
This fact will be important in my paper \cite{heinemann2012cannot}.

The relation between the Bethe partition function \eqref{eq:bethe_approximation} and the exact one \eqref{eq:variation_A} is still an open question.
It is clear that a relation cannot be given in the general case - the Bethe approximation is sometimes bigger and sometimes smaller than the exact partition function.
One of the first results was by \cite{AlanNips2007}, proving that if all the pseudo-marginals are in the same orientation (all bigger or smaller than $0.5$), then the Bethe partition function lower bounds the exact partition function.
Later, \cite{RuozziNips2012} gave a stronger result, showing that if the pairwise interactions are log sub-modular, then the Bethe partition function lower bounds the exact partition function (the case of attractive models is an instance of this class).
In their proof, they used the important work of \cite{vontobel2013counting}, in which a combinatorial meaning was given to the Bethe entropy.

To conclude, the connection between BP and Bethe contributes to the understanding of BP and helps improve it.
The biggest shortcoming of this connection is that it says nothing with respect to BP messages before conversion.
Another interpretation of BP is seeing it as a re-parametrization algorithm, that tries to change the parametrization of $\thetav$ such that it will belong to the local-polytope.  
For full proof please refer to \cite{wainwright2002stochastic}.
\ignore{
\be
\mu_k(x_k;\thetav) = \frac{1}{Z(\thetav)}\sum_{\substack{\xx \\
s.t.\  \xx_k=x_k}}e^{\theta_k(x_k) + \sum_{j \in \nei{k}}\theta_{k,j}(x_k,x_j)}e^{\sum_{i \in V \setminus k}\theta_{i}(x_i) +\sum_{\substack{ij \in E\\
 s.t.\  i,j \ne k}}\theta_{ij}(x_i,x_j)}
%} {\sum_{\hat{x}_k}e^{\theta_k(\hat{x}_k) + \sum_{j \in \nei{k}}\theta_{k,j}(\hat{x}_k,x_j)}e^{\sum_{i \in V \setminus k}\theta_{i}(x_i) +\sum_{\substack{ij \in E\\
% s.t.\  i,j \ne k}}\theta_{ij}(x_i,x_j)}}
\ee
Denote by $\thetav^{\setminus k}$ the model where we remove all factors involve the vertex $k$.
Now the marginal of the neighbors of $k$ in  that model is
\be
\muv_{\nei{k}}(\xx_{\nei{k}}; \thetav^{\setminus k}) \approx \sum_{\substack{\hat{\xx}\\
s.t. \hat{\xx}_{\nei{k}} = \xx_{\nei{k}}}}  e^{\sum_{i \in V \setminus k}\theta_{i}(\hat{x}_i) +\sum_{\substack{ij \in E\\
 s.t.\  i,j \ne k}}\theta_{ij}(\hat{x}_i,\hat{x}_j)}
\ee
 With this we can write
\bea
\mu_k(x_k;\thetav)  &\approx& \sum_{\xx_{\nei{k}}} e^{\theta_k(x_k) + \sum_{j \in \nei{k}}\theta_{k,j}(x_k,x_j)} \muv_{\nei{k}}(\xx_{\nei{k}}; \thetav^{\setminus k})\\
 &\approx& e^{\theta_k(x_k)}  \prod_{j \in \nei{k}} \sum_{ x_j } e^{\theta_{k,j}(x_k,x_j)} \muv_{j}(x_j; \thetav^{\setminus k})\\
\eea
}
\subsection{Sampling}
\label{sec:sampling}
At first glance, sampling seems like the perfect solution for finding the marginals.
The marginals may be estimated as the mean of bounded random variables.
Using the Hoeffding bound, the probability of mistake greater than $\epsilon$ can be bound by $P(|p(x_i) - \frac{\sum_{m=1}^M \deltaF{\xx_i^m=x_i}}{M}| > \epsilon) < 2\exp{-\frac{2\epsilon^2M}{|\cX|^2}}$; meaning, the time complexity is $O(\frac{\log(\delta) |\cX|^2}{\epsilon^2})$\footnote{With a probability greater than $1-\delta$, the estimation error will be less than $\epsilon$.}.
The problem in this is, that sampling from a model as in \eqref{eq:ciluqe_prob} is hard in itself\footnote{This can be easily seen from redction to q-coloring, for example \cite{levin2009markov, bordewich2016mixing}}.
How, then, can a model be sampled?

Markov Chain Monte Carlo (MCMC) is a method of sampling from a distribution $P(\xx)$ while being able to sample only from distribution $Q(\xx)$.
In this, a Markov chain with two main features is constructed: first, it must be easy to move from one state to another; secondly, the stationary distribution of the Markov chain is $P(\xx)$.

To define a Markov chain, one needs to define the transition matrix between any two states.
The matrix rows should sum to one $\sum_j A_{i,j} = 1$, the coordinate $A_{i,j}$ defining the probability to move from state $i$ to state $j$.
In MCMC, this matrix of dimensions is $A \in [0,1]^{|\cX|^{p},|\cX|^{p}}$, meaning that each state is a full assignment of the model variables. 
The stationary distribution $\vv$ is a distribution over the states (over different assignments) that satisfies $A\vv=\vv$.
In MCMC, this distribution should be $P(\xx)$ - the model distribution.

Metropolis chains is an example of how the Markov chain can be constructed.
The transition matrix is defined as follows:
\be
A(\xx,\tilde{\xx}) = \left\{
\begin{array}{lr}
Q(\tilde{\xx}|\xx)\min\{1,\frac{Q(\xx|\tilde{\xx})P(\tilde{\xx})}{Q(\tilde{\xx}|\xx)P(\xx)}\} & \tilde{\xx} \neq \xx\\
1 - \sum_{\hat{\xx} \neq \xx} Q(\hat{\xx}|\xx)\min\{1,\frac{Q(\xx|\hat{\xx})P(\hat{\xx})}{Q(\hat{\xx}|\xx)P(\xx)}\} & \text{else}
\end{array} \right.
\ee
Note that $Q(\xx|\tilde{\xx})$ must only be irreducible\footnote{Irreducible defined: for all $\xx,\xx_0 \in \cX^p$ exists $k \in \naturalNumbers$ and $\xx^1, \ldots, \xx^k=\xx$, such that $\prod_{i=1}^k Q(\xx_i|\xx_{i-1}) >0$} - it does not necessarily have to be symmetric; hence, it could be the distribution over a tree or any easy to sample distribution.
Moreover, we need only to compute $\frac{P(\xx)}{P(\tilde{\xx})} $, hence the partition function is canceled, and the quantity is easy to calculate.

Glauber chains, or Gibbs sampler, is another method to define the Markov transition matrix.
The idea is simple, in each iteration only a small number of variables will change, this group denoted by $\cI \subseteq [1,\ldots,p]$, $|\cI| = k$.
Then, the possible next states are the ones that differ from the current state $\xx$ only in the indices in $\cI$, this set denoted by $\mathcal{D}(\xx,\cI) = \left\{\tilde{\xx} \in \cX^p | \forall i \notin \cI,\ \xx_i = \tilde{\xx}_i \right\}$.
The set $\mathcal{D}^k(\xx)= \cup_{\cI, |\cI|=k} \mathcal{D}(\xx,\cI)$ denotes the set of all $\tilde{\xx}$ that differ from $\xx$ only in $k$ entries.
Now the transition matrix is defined as,
\be
A(\xx,\tilde{\xx}) = \left\{
\begin{array}{lr}
0 & \tilde{\xx} \notin \mathcal{D}^k(\xx)\\
\frac{(p-k)!k!}{p!}\frac{P(\tilde{\xx})}{\sum_{\hat{\xx} \in \mathcal{D}(\xx,\cI)}P(\hat{\xx})} & \tilde{\xx} \in \mathcal{D}^k(\xx), \xx \neq \tilde{\xx}\\
\frac{(p-k)!k!}{p!}\sum_{\cI, |\cI|=k}\frac{P(\xx)}{\sum_{\hat{\xx} \in \mathcal{D}(\xx,\cI)}P(\hat{\xx})}& \xx = \tilde{\xx}
\end{array} \right.
\ee
Note that as in the case above, only the ratio of $P(\xx)$ needs to be computed, hence the partition function can be canceled.

The remaining question is, how fast are we guaranteed to sample from the stationary distribution when starting from an arbitrary assignment $\xx$.
This constant is called the \textit{mixing time}.
The mixing time is defined as the maximum number of moves (transitions between states) necessary until the visited states distribution is $\epsilon$ close to the stationary distribution while starting from any state.
\be
\tau_{\epsilon}  = \min_{\tau \in \naturalNumbers}\sup_{\vv \in \Omega} \left\{\tau | d\left(A^{\tau}\vv,P(\xx)\right) < \epsilon\right\}
\ee
where $\Omega = \{ \vv \in [0,1]^{|\cX|^{p}} | \sum_i \vv_i = 1\}$, and $d(\vv,\tilde{\vv})$ is any distance function between probabilities.\footnote{ From the above condition it is easy to see the connection to the spectral gap (the distance between the largest to the second eigenvalue) and the mixing time. For more information see\cite{levin2009markov}}  .
The mixing time depends on the model features, such as interaction strength, dependency structure etc.
Hence, models with a small mixing time can be easily sampled, and inferring the marginals is doable.
This fact will be important when talking about inferning \secref{sec:inferning}.

The next section covers learning graphical models.

